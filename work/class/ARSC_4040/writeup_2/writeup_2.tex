\documentclass[12pt]{article}

\title{Summary and Reflection on ``Cooperative Learning Returns to College''}
\author{Luke Palmer}

\begin{document}
\maketitle

``Cooperative Learning Returns to College'' by Johnson \textit{et
al.} presents a strong case for cooperative learning along with some
techniques for its implementation.  

The paper begins by defining cooperative learning where ``students can
work together cooperatively to accomplish shared learning goals''.  A
defining characteristic of cooperative learning, as opposed to
collaborative learning or competitive learning, is that of
interdependence.  Each student satisfies his learning goal if and only
if all other group members satisfy theirs.  

The authors then go on to identify the theoretical roots of cooperative
learning, giving evidence that cooperative learning is beneficial
theoretically from the standpoints of social interdependence theory,
cognitive-developmental theory, and behavioral learning theory.  The
authors also developed controversy theory, which states that strong
opposing points of view in a group, after being well argued, will result
in a ``thoughtful and refined conclusion''.

We then get to the section ``The Internal Dynamics That Make Cooperation
Work'', which was the most interesting to me.  Here the authors list
five guidelines for implementing a successful cooperative learning
environment.  I'll reflect on each of these guidelines with respect to
the physics tutorials:

\textbf{Ensure that each student perceives that he or she is linked with
others in such a way that the students cannot succeed unless the others
do}:  This was not really the case in the physics tutorial.  In fact,
there were no explicit learning goals within the groups, especially in
the short-term class period.  Students are supposed to fill out the
tutorial book, so we stuck them at tables and said ``work together''.
Many times they did, often explaining to each other, but sometimes they
would just sit at the table and each would work on his own.

\textbf{Structure individual accountability so that the performance of
each student is assessed}: Introducing a grade for the tutorial books
would be a bad idea, as the students would feel pressured for time, and
thus they wouldn't take the time to explain things they understood to
other students (an almost competetive learning symptom).  I entertained
the idea of having a monthly quiz where each group member's score is
tied to each other's.  That might also create a competetive and
unrelaxed atmostphere, insighting hostility toward the students who do
not grasp the concepts as clearly.

\textbf{Ensure that students promote one another's success face to
face}:  I think the best way to accomplish the three goals stated so far
would be to restructure the tutorial sessions.  Instead of having each
student fill out a fairly easy series of questions along with their
classmates, perhaps the tutorials should work with one experiment with a
result that is not so trivial (or one that the book doesn't just tell
them after awhile).  The trouble with that is that the recitations only
last an hour, which is generally not enough for students to really
\textit{discover} something.  I was surprised about how much
encouragement was already going on in the tutorials, even with no
external structure to facilitate it.  Perhaps simply the low pressure
environment and the common (ungraded) goal is enough to promote mutual
encouragement.

\textbf{Teach students the needed social skills and ensure that they are
used appropriately}:  This was informally done by the learning
assistants, but generally it was not done at all.  The best we ever
tried was ``can you explain that to him?''.  It didn't seem to be an
issue, however there was mostly no leadership within the groups---the
tutorial book led, the students followed.  Since group skills are
important, this is another advocate for a different tutorial structure,
where leadership and group decision-making are necessary.

\textbf{Ensure that students take the time to engage in group
processing}: Another guideline which was omitted from the tutorial
structure.  However, this could be ``hacked'' in easily (I'll likely
suggest this for next year).  Have students every month or so reflect to
each other about their roles in the group.  Since face-to-face
derogatory discourse makes people uncomfortable, the students would
likely refrain from saying that (even if they felt it), while still
personally reflecting on themselves and others.  This could also help to
achieve guideline \#3.

In conclusion, the tutorials in which I worked had a lot of cooperative
learning guidelines implemented, but there are some areas that it would
be worth experimenting with.  Next year in my Calculus 1 learning
assistantship, where the learning assistants (apparently) have more
freedom in creating the sessions, we will think back upon these
guidelines and see what sort of structure we can create to implement
them.

\end{document}

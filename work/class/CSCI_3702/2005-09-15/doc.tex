\documentclass[12pt]{article}

\usepackage{setspace}

\begin{document}
\noindent Luke Palmer \\
2005-09-15 \\
CSCI 3702 \\
\textbf{\large{Moutoussis and Zeki: The relationship between cortical activation and
perception investigated with invisible stimuli.}}

\onehalfspace

This paper described an experiment in which people were shown ``inverse
images'' to each eye, so that the perceived image vanished.  For
instance, a green house on a red background in the left eye, and a red
house on a green background in the right eye.  The viewer conciously
perceives a plain yellow plane.  The experimentors measured brain
activity while the subjects looked at this yellow ``wall'', and found
that the same regions were active as when a subject was viewing the
house conciously, but to a lesser degree.

One might conclude from this paper (as the experimentors implied) that
conciousness is determined by the \textit{level} of activity in the
appropriate brain areas, since unconcious awareness of a house or a face
resulted in the same active areas of the brain.

I personally wonder whether one could evoke emotions without concious
input; that is, could you show a subject a picture of his wife, or a
dying child, and make the subject feel a certain way but not understand
why?  Unfortunately, it would be difficult to perform an experiment of
this nature, since it is difficult for subjects to objectively report on
their emotions.

\end{document}

\documentclass[12pt]{article}

\begin{document}
\noindent
Luke Palmer \\
2005-11-10 \\
Chang et al.  \textbf{Structural Priming as Implicit Learning: A
Comparison of Models of Sentence Production.}

\vspace{1cm}

This paper presents a study showing evidence for \textit{structural
priming}, that is, the fact that recently heard sentence
structures---independent of content---are likely to be used again. The
experimenters created a model that showed slightly better priming
effects than previous models.  They believed that the model was superior
because of one crucial aspect, however:  it exhibited
locative-to-passive priming\footnote{Priming from ``dogs are walking''
to ``cats are chased''.}.  The model used comprehension to guide
production of future sentences, and incorporated nonatomic message
roles.

The use of nonatomic message roles seems to be the most significant
result of this study; that is, the usage of (Source, Theme, Goal)
instead of (Agent, Patient, Goal, Location, ...).  This is just lifting
one level of meaning out of structure, adding a level of indirection, as
it were.  This gave much better results than atomic message roles, which
are more tightly coupled to sentence structure.  Is there another level
of indirection that can be abstracted out?

\end{document}

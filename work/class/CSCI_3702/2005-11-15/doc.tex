\documentclass[12pt]{article}

\begin{document}
\noindent 
Luke Palmer \\
2005-11-15 \\
\textbf{Linguistic Society of America FAQ.} \\
Majid et al.  \textbf{Can language restructure cognition?  The case for
space.}

\addvspace{1cm}

After having a background in procedural programming languages like C++
and Perl, I remember venturing out to learn Haskell because of an open
source project I wanted to work on.  Haskell is a completely different
language family from the other languages I knew; specifically, the
functional family.  But unlike Lisp and Scheme (which I had experience
with), it completely forbids the programmer from falling back on
procedural methods.

At this point, I could learn a new language in a day or two.  Thus, I
found it deeply disturbing when it took more than \textit{two months} to
learn Haskell.  I spent hours and hours programming something as simple
as a tic-tac-toe game.  However, after a while, I started to get used to
it.  My speed increased, I began answering questions others asked about
it, and I finally became able to work on the project.

After a while, I went back to program a few things in the languages I
already knew.  However, my style of programming had dramatically
shifted.  I was thinking about problems completely differently---the
Haskell way.  I abstracted things differently, factoring out algebraic
structures instead (or in addition to) related data. 

I have a different answer for the FAQ question: ``So learning a
different language won't change the way I think?''  Yes, it will.  In
order to even be coherent in Haskell, I had to rewire my brain.  Our use
of language is equivalent to our ability to represent
\textit{abstractions}.  If you learn a language that makes use of
abstractions you are not familiar with, you will have to change the way
you think to speak in that language.  It is not much of a stretch for an
English speaker to learn Spanish or German, because English is a
descendent of both of those families.  But for an English speaker to
learn Arapaho is going to take some rewiring.

Frame of Reference is just one such abstraction, but it seems to be one
that typically varies from family to family.  In Hawaiian, an absolute
language, the absolute directions are ``towards the center'' and ``away
from the center'' of the island.  It seems that there are no specific
environmental advantages to abstracting one way or the
other\footnote{You could argue that absolute is less advantageous
because it involves keeping track of your current orientation at all
times; but then you could also argue that relative is less advantageous
because you need to constantly calculate frames of reference.  The truth
is, it's only hard if that's not how you're wired to think.}.  However,
it does make sense that a language as a whole would make this
distinction, as opposed to changing on an individual basis, because you
wouldn't want to talk to your friend and say ``it's to the north of the
phone'' and have your friend take a minute to calculate which way is
north from the phone.

Another abstraction that was passively mentioned in the Majid et al.
paper was ``structure-mapping''; i.e. the method by which speakers pull
out higher-order similarities in structures.  This is very much akin to
the change in my thought process after learning Haskell.  The structures
the language provides for abstraction affects the way you think about
abstraction (similarly, the abstractions that are useful in your
environment will make their way into the language).

\end{document}

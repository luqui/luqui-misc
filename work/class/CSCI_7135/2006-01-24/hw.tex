\documentclass[12pt]{article}

\begin{document}
\noindent 
Luke Palmer \\
CSCI 7135 \\
2005-01-24

PAPI, the ``Performance Application Programming Interface'' is a
platform-independent\footnote{When you want it to be.} interface to the
hardware performance counters of many architectures.  It provides its
users with two views, what it calls the ``high level'' and ``low level''
views.   The high level view provides a simple interface that is totally
platform-independent, providing common counters available everywhere.
The low level view allows specification of hardware-specific counters,
as well as a few ``advanced'' features, such as counter multiplexing.

PAPI's interface is fairly straightforward: create an event set
describing the events to monitor, ask it to start monitoring those
events, then ask it what it got.  PAPI provides an abstraction called
\textit{multiplexing} to monitor more events at a time than the
processor is capable of, which conveniently has the same interface as
non-multiplexed monitoring, except that you may only use the low level
interface ``to prevent na\"ive use'' (ugh).  Multiplexing is implemented
by giving each monitor a time slice, much like the processor threads
processes.  This is done in software, so switching the monitors too
often could have an effect on the performance profile of the program.

Perhaps the most powerful feature of PAPI is its \textit{statistical
profiling}.  This interrupts the program in regular intervals and
records the value of a monitor in bins indexed by the program counter.
This gives a statistical appoximation to how much each portion of the
program triggered that event.  However, exactly how often to interrupt
is a difficult question: if we interrupt too frequently, writing the
results to memory could skew results regarding cache.  If we interrupt
too infrequently, the program will have to be run for a long time to
gather accurate results.

That doesn't leave much to say.  PAPI looks to be a well-designed
(with the exception of the ``na\"ive use'' thing above) and full
featured library.

\end{document}

\documentclass[12pt]{article}

\begin{document}
\noindent
Luke Palmer \\
CSCI 7135 \\
2005-01-31

With the exception of a few key features, the Pentium 4's philosophy
seems to be ``make everything bigger''.  The Pentium 4 has a longer
pipeline (to allow higher clock speeds), a larger L2 cache and data path
width, and more internal registers than the Pentium 3.  Despite the
appearance of making this the SUV of processors, the papers argue that
this does have real benefits due to hyperthreading.  

A new technology, hyperthreading, is supported on newer models of the
Pentium 4.  Hyperthreading allows running two-threaded programs without
context switching, by keeping two states (registers, PC) in the
processor simultaneously, but sharing most resources, such as data
cache.  Increasing the size of some resources has little effect on
single-threaded programs, but a large effect on multi-threaded programs.

Pentium 4 also supports the ``trace cache'', which is an instruction
cache that caches instructions \textit{after} they have been decoded
into microcode.  It also orders the cached $\mu$ops into sequences (or
``traces''), so that one form of branch prediction comes for free; i.e.
the destination of a recently taken branch will be included in the same
trace as the source.

Finally, Pentium 4 flaunts an out-of-order core (a feature not so
uncommon these days) which tries to keep every subunit of the processor
busy as much as possible.  A memory load, a floating point multiply, and
an integer shift can all be performed simultaneously by queueing
instructions of various types for their respective cores.  In order to
support this out-of-order execution, Pentium 4 does ``register
renaming'' to avoid working on corrupt data.  I'm not sure how that
works.

Intel is phasing out the Netburst architecture, going back to the
Pentium III architecture for their next generation of processors, the
Pentium M.  Although the Pentium 4 could achieve incredibly high clock
frequencies, the long pipeline must have punished branch mispredicts too
much.  A lot of modern software has many, many branches (consider
virtual machines with null pointer checks or vtable dispatches), so
optimizing for ``pure computation'' flow was probably a poor choice.
Then again, the Pentium 4 performs well with multimedia applications:
video encoding and playback for example.
\end{document}

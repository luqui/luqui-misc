\documentclass[12pt]{article}

\begin{document}
\noindent
Luke Palmer \\
CSCI 7135 \\
2006-02-07

This paper presented the design decisions and the features of Intel's
Pentium M processor for mobile computers.  An analysis showed that a 3\%
power increase could yield a 1\% performance increase by clock scaling.
This was used as a baseline metric for determining which features to
include in the Pentium M.  If the ratio between the power increase
and the performance gain was less than three, the feature was, in
general, included.

Because a processor takes about 10\% of the power used in a laptop
computer, the power optimization was made to reduce heat, not
necessarily to increase battery life.  However, the paper gave little
more than a mention of the ``Power Density'' constraint.  Given that
battery life is not the goal, this issue deserved more treatment.

The Pentium M features a smarter branch predictor in order to reduce the
number of speculative instructions executed.   Notably, it can predict
indirect branches which is a win for object-oriented software, or any
software running in a virtual machine, which is becoming more and more
common.  The Pentium M also has ``micro-ops fusion'', which splits a
single memory $\mu$op only at the execution level to reap the benefits
of out-of-order execution, rather than splitting it into two completely
separate $\mu$ops.  It also flaunts an interesting (but confusing)
technology related to the stack pointer.

Finally, the paper includes a benchmark analysis comparing the Pentium M
to the Pentium 4 and Pentium III processors.   The statistical analysis
fell short of being thorough and rigorous, but glancing at the charts
looked favorable, as one would expect.

\end{document}

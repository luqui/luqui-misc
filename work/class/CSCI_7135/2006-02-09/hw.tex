\documentclass[12pt]{article}

\begin{document}
\noindent
Luke Palmer \\
CSCI 7135 \\
2006-02-09

This Tom's Hardware article contrasted\footnote{Accidentally, so they
say.} the performance and power usage of the Pentium M with the Pentium
4.  The article tested the Pentium M as if it were a desktop processor,
using the Asus CT-479 adapter, which allows a Pentium M to be placed in
a Pentium 4 chipset.  They performed a series of speed and thermal
performance benchmarks, finding that the Pentium M often outperforms the
Pentium 4 even as a desktop processor.

The authors found that it was possible to overclock the Pentium M up to
2.56 GHz by overclocking the front-side bus.  The overclocked Pentium M
was at the top of the list (against a selection of Pentium 4s and Athlon
64 FXs).  The Pentium M didn't beat the Pentium 4 on video encoding
tasks, however; it lagged behind by a fair amount.  We've seen that the
Pentium 4 is good at these kinds of tasks because of the trace cache and
SSE3, neither of which the Pentium M supports. The Pentium M also didn't
outperform the Pentium 4 on the memory benchmarks.  This is most likely
because of the 160 MHz bus, whereas the top-performing Pentium 4 had a
266 MHz bus at its disposal.

However, the Pentium M has better power consumption (and therefore heat
dissipation) by 80\% (at peak load).  The fact that it outperformed the
Pentium 4 in many cases combined with the power win caused the authors
to question why the Pentium 4 was even necessary.  They stated that
Intel planned to phase out the Pentium 4.  In light of these results,
this seems like a wise move.

One thing that I noticed when reading the Pentium 4 and Pentium M
articles previously was that the Pentium 4 flaunted a bunch of new
features like a \$29.99 late-night infomercial.  The Pentium M instead
included a detailed analysis of all the features that could be included,
and a precise metric determining whether they should be.  This shows
that it pays to do engineering when engineering a processor.

\end{document}

\documentclass[12pt]{article}

\usepackage{amsmath}
\usepackage{amssymb}
\usepackage{amsthm}

\DeclareMathOperator{\eval}{eval}
\DeclareMathOperator{\dom}{dom}

\newtheorem{lemma}{Lemma}
\newtheorem*{theorem*}{Theorem}
\newtheorem*{corollary*}{Corollary}

\newcommand{\arr}[0]{\rightarrow}

\begin{document}
\noindent Luke Palmer \\
ECEN 5023 \\
Homework 1

\begin{description}
\item[(1)]
 Define the family of relations $(\arr^k)$ as follows:

 \begin{itemize}
  \item $x \arr^0 y$ iff $x = y$.
  \item $x \arr^1 y$ iff $x \arr y$.
  \item $x \arr^{k+1} y$ iff there is a $z$ such that $x \arr^k z \arr y$.
 \end{itemize}
   
 \begin{lemma}
  \label{lemma-partialfunction}
  For any $k$, $(\arr^k)$ is a partial function.
 \end{lemma}
  \begin{proof}
   By induction on $k$.  $k = 0$ is trivial. 

   If $a \arr^{k+1} b$ and $a \arr^{k+1} b'$, then there exist $z,z'$
   such that $a \arr^k z \arr b$ and $a \arr^k z' \arr b'$.  By the
   induction hypothesis $z = z'$, and by the determinacy of single-step
   evaluation $b = b'$.
  \end{proof}
 
 \begin{lemma}
  \label{lemma-normalform}
  If $t \arr^k u$ and $u$ is in normal form, then $t \not\arr^n x$ for
  any $n > k$.
 \end{lemma}
  \begin{proof}
   When $n = k+1$, $t \arr^k z \arr u$.  But by Lemma
   \ref{lemma-partialfunction}, $z = u$, and since $u$ is in normal
   form, $u \not\arr u$, a contradiction.

   When $n = k+a+1$ ($a>0$), $t \arr^{k+a} z \arr u$, but by the induction
   hypothesis, $t \not\arr^{k+a} z$ in the first place.
  \end{proof}

 It is straightforward to show $(\arr^*) = \bigcup\limits_{i\in\omega}{(\arr^i)}$.  

 \begin{theorem*}
  (Uniqueness of Normal Forms) If $t \arr^* u$ and $t \arr^*
  u'$, where $u$ and $u'$ are both normal forms, then $u = u'$.
 \end{theorem*}
  \begin{proof}
   There exist $k,l$ with $t \arr^k u$ and $t \arr^l u'$.  Wolog suppose
   $k \le l$.  But $u$ is in normal form, so by Lemma
   \ref{lemma-normalform}, $t \not\arr^n x$ for any $n > k$.  Therefore
   $k = l$, and since $(\arr^k)$ is a partial function, $u = u'$.
  \end{proof}

\item[(2)]
  \begin{theorem*}
   (Progress) For every term $t$, either $t \in \{\mathtt{true},
   \mathtt{false} \}$ or there exists a $t'$ with $t \arr t'$.
  \end{theorem*}
   \begin{proof}
    By structural induction on $t$:
    \begin{itemize}
     \item \texttt{true,false}: We're done.
     \item \texttt{if true then $a$ else $b$}: $t \arr a$ by
     \textsc{E-IfTrue}.
     \item \texttt{if false then $a$ else $b$}: $t \arr b$ by
     \textsc{E-IfFalse}.
     \item \texttt{if $c$ then $a$ else $b$} where $c \not\in
     \{\mathtt{true}, \mathtt{false}\}$: By the induction hypothesis,
     there is a $c'$ with $c \arr c'$, so $t \arr $ \texttt{if $c'$ then
     $a$ else $b$} by \texttt{E-If}.
    \end{itemize}
   \end{proof}

\item[(3)]
   There is no Progress theorem for the full language.  For example,
   \texttt{succ true} is not a value but there is no rule that applies
   to it.

\end{description}
\end{document}

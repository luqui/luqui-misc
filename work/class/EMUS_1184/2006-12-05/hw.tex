\documentclass[12pt]{article}

\begin{document}
\noindent Luke Palmer \\
2006-12-05 \\
EMUS 1184

\begin{center}
\large{What have I learned in voice class?}
\end{center}

The most beneficial thing about the class is the most difficult to put
into words.  The biggest factor in learning to become a musician is
experience, and I will be walking out of this class with a semester's
worth of experience with my voice and with an expert who helps me
improve faster.  That's what I value in this course.

My friend Colette once said, ``every teacher should always be learning a
new foreign language or a new musical instrument in order to remind him
what it's like to feel completely stupid.''   I haven't been a beginner
musician in a long time, so this has been an interesting experience.  I
had forgotten (or never knew) the \textit{incredible} value of
practicing.  Starting a new piece sometimes feels like just learning to
drive; there is way too much to keep track of.  Working for just an hour
with a piece adds enough solidity and comfort to be able to bring out
one aspect and work on it, without worrying about the others.

I learned about the power of control over breath.  Before this class, I
had always thought about tone and the shape of the mouth being the two
most important things about voice.  Now I see that from powerful,
directed, and controlled breath the other aspects follow quite easily.
Determining beforehand where and how to breathe in a piece makes its
performance flow much more easily.

The effectiveness of small corrections in music never ceases to amaze
me: whether it is the use of the pedal or accentuating stacatto on my
piano, or reshaping a vowel or articulating consonants a little more in
my voice.  I'm still not quite sure what to do when in that area; I need
Lucas's guidance.  But I think that will come with experience.
\end{document}

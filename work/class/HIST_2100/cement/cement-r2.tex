\documentclass[12pt]{article}

\usepackage{setspace}

\begin{document}
\noindent Luke Palmer \\
2005-11-09 \\
HIST 2100

\begin{center}
\Huge{Cement}
\end{center}

\doublespace

Among the major themes in the novel there are three which characterize
the attitudes and values of the 1920s in Russia.  The first is the
attitude that no exploiter is a good Communist.  Second, that the
individual suffers for the good of the whole.  Finally, that there is a
strong unification among the Communists, there is no sense of personal
relationship (but this may just be an attitude of Mekhova).

Dasha is a strong representative for the attitude that Communists must
not be exploiters.  For the first half of the book, she continues to
represent to Gleb the fact that she will not be exploited by him.  ``But
you are also a brute man, needing a woman to be a slave for you, for you
to sleep with.  You're a good soldier, but in ordinary life you're a bad
Communist.''   

Zhidsky and Tskheladze feel that Badin is committing a scandal and
ruining the Party by meeting in Shramm's high-class room, indeed,
exploiting a maid.  Badin is ``trying to insinuate himself into party
ranks'', a phrase that shouldn't even be sensical in a Communist
community. Such secretive pleasures should be no one's; they all share
the burden of work and the fruits that are reaped.

Dasha and Gleb both provide us with a good example of the value that the
good of the movement justifies the individual's suffering. Gleb gives a
speech to the workers stating that ``they have won this day and that
with their work and suffering, they can win the world.''  Dasha
sacrifices the care of her own daughter, which will mean her daughter's
life, in order to concentrate on work and the movement.  Work comes
before everything: ``their hearts must be of stone,'' she says.

Mekhova represents the theme that, while the Communists are unified and
together in cause, they have no personal relationships.  ``That's just
it,'' she says, ``we're strongly organized, strongly bound together.  But
we're terribly apart, one from another, in our private lives.''  This is
another facet of the sacrifice theme, where one must exchange love and
family for work and the movement.  Mekhova continues to question this
value, both in her words and her actions, causing her to appear weak.

\end{document}

\documentclass[12pt]{article}

\usepackage{setspace}

\title{Vera Figner}

\begin{document}
\noindent Luke Palmer \\
2005-09-09 \\
HIST 2100

\begin{center}
\Huge{Vera Figner}
\end{center}

\doublespace

Vera Figner's reasons for becoming a revolutionary and a terrorist are
tied to the intellectual and emotional influences she met in Zurich, and
the contrast she felt when she returned to Russia.

Vera traveled to Zurich in order to attend the university and study
medicine, which she intended to use to treat the peasants.  There she
attended a short-lived women's club, which led her to join a student
club called the Fritsche.  This club would heavily influence Vera's
future political choices.

In Zurich, Vera was exposed to ideas that had never even occurred to her
during her upper-class Russian upbringing.  She developed a moral
conflict between her growing belief in the the ideas that ``a small
group of people at the top [are] responsible for ... the sufferings of
the laborers in industry and in workshops everywhere'' (14) and the
realization that she was one of those people on top. 

\begin{quote}
\singlespace
I never asked myself how such a system could be established in
Russia---the question never even occurred to me.  My uncle had once told
me ... ``Every nation gets the government it deserves.'' (12)
\end{quote} 

As her intellectual development continued in Zurich, she began to see
that a doctor who helps a few hundred people would do nothing to help
the poor compared to a change in social structure---a change in
government.  This was the way to create equal well-being for all people.

Vera was subject to several emotional influences in Zurich.  She was
captivated by the heroic contrast between her own sad fate and the
people's bright future.  Her social circles were certain of the
inevitability of social revolution, while this idea was under harsh
persecution in Russia.  ``If these ideas had not been subject to
persecution in Russia, then it might have been possible to examine
disagreements or doubts about them on their merits.'' (18)

Vera Figner became a revolutionary because of the discomfort she felt in
her old beliefs and the freedom and acceptance she enjoyed in her
revolutionarily-minded social group. 

The Fritsche eventually formed the nucleus of the All-Russian Social
Revolutionary Organization, which ventured to Russia to work among the
peasants and spread the idea of revolution (which was a miserable
failure).  Vera stayed back for a year, but eventually abandoned her
diploma and returned to Russia as well.  

She settled in the countryside to become a paramedic in a
\textit{zemstvo}.  There she was subject to unending persection. ``I
lived in an atmosphere of suspicion.'' (41)  She became convinced that
her struggles there were due to the absence of political freedom in
Russia.

Vera felt first-hand the lack of freedom---the oppression---in Russia,
and decided that becoming a terrorist was the only way to achieve her
revolutionary goal. ``My past experiences had convinced me that the only
way to change the existing order was by force.'' (43)  The government
and the culture would simply smother any other way.   

\end{document}

\documentclass[12pt]{article}

\begin{document}
\noindent
Luke Palmer \\
LING 2000 \\
2006-02-10

Ten examples of conceptual metaphor.

\begin{itemize}
\item From ``Whose Line is it Anyway?'': ``You're going to sing snippets
of the songs.''

You can't really snip a song, so there can't be a snippet of a song.
``You're going to read snippets of paper'' is a literal utterance of the
same form.

\item From ``South park'': ``Go to your homes and arm yourselves with
whatever you can.''

This concerns the form ``arm'' meaning to prepare the use of a weapon.
Its etymology probably has to do with putting the weapon on your arm.
It may not be a metaphor, because the word ``arm'' as a verb has no
other meaning in common use.

\item From ``Family Guy'': ``We miss you.''

This is a derivative of ``we missed you'', which is itself a metaphor.
It compares sight or meeting to throwing something or firing a weapon.
``We threw the rock, and we missed you'' is an example of a literal use.

\item From ``Slashdot'': ``After the notorious JPEG patent which has
made many big and small names pay huge amounts...''

Neither names nor amounts can be literally huge, big, or small.
``...which has made many big and small people give huge gifts'' is an
example of a literal use (it probably would not be interpreted
literally, but it can be).

\item From ``Slashdot'': ``Oracle plans to cut 2,000 jobs across the
Siebel and Oracle work forces...''

This treats a job as, for example, hanging by a string.  To cut the job
means to cut the string on which it hangs.  ``Oracle plans to cut 2,000
papers into small pieces'' is a literal use.

\item From ``Slashdot'': ``As a thank-you, we are giving MacBook Pro
computers to twelve of our top contributors.''

Here, ``top'' is using the common ``up is good, down is bad'' metaphor.

\item From ``Slashdot'': ``...warning that a bad 2006 could force the
former high-flyer into bankruptcy.''

This is another example of ``up is good, down is bad''.  The high-flyer
is a company that is doing well.  A literal use: ``...warning that a bad
storm could force the former high-flyer onto the ground'' (speaking
about a pilot).

\item From ``Slashdot'': ``Turns out popularity bred popularity...''

Popularity cannot literally breed.  ``Turns out the farmer bred a
seedless grape'' is a literal use.

\item From the Pugs development blog \texttt{pugs.blogs.com}: ``Lexical
imports and rich module interface information in \texttt{\%*INC}''

``Rich information'' does not literally have a lot of money.  ``Rich''
is being used as ``much.''  ``Rich businessman'' is an example of
literal use.

\item From Dan Sugalski's blog \texttt{www.sidhe.org/\~dan/blog}:
``...since thinking about APL tends to make me think of Unicode, and
when I do that I need a good lie-down until it passes''

This is talking about an implicit illness or woosiness ``passing'', from
the same metaphor used in ``there is a cold going around''.  A literal
use: ``I need a good lie-down until the supersonic jet passes.''

\end{itemize}
\end{document}

\documentclass[12pt]{article}

\begin{document}
\noindent Luke Palmer \\
2006-04-21 \\
LING 2000

\section*{10-10}

\begin{description}
\item[a.] Adjectives are extremely common compared to conversation.
Every other category seems to be drowned out by the adjectives.

\item[b.] A list of adjectives forms a sentence.  This list of
adjectives may be followed by a noun, but no verb.  Such sentences
describe the writer of the ad (except in case (1) where ``You:'' is used
to list desired adjectives of the reader).  The writer speaks about
himself in the third person.  In sentences where the writer is the
subject, the subject is left out.  In such sentences where the verb is
in the progressive form, the auxiliary ``is'' is also left out (eg.
``Seeking counterpart, ...'').  Most ads include an acronym of the form
[SG][WDB][MF], where the letters stand for Straight, Gay, White, Dark?,
Black, Male, Female respectively.  Most verbs are in the third person
singular.

\item[c.] Seeks (third-person singular), respects (TPS), lives (TPS),
works (TPS), enjoys (TPS), seeking (TPS?), likes (TPS), looking (TPS?).

\item[d.] \textit{He is a} slim, young gay white male \textit{who has a}
very straight appearance, \textit{is} masculine, athletic, healthy,
clean-shaven, \textit{and} discreet.  \textit{He} seeks \textit{a}
similar good-looking white male \textit{who is} under 25 for \textit{a}
monogamous relationship.

\item[e.] It was possible to fill in words to make this essentially
conversational English, as long as it is in the third person.

\item[f.] The acronyms GWM, DF, etc. Abbreviations such as n/s.
\end{description}

\section*{11-1}

Most of the time, a dialect is defined to be a mutually intelligible
form of a language.  That is, two people who speak different dialects of
the same language can understand each other, whereas they can't if they
speak two different languages.  However, politics sometimes gets in the
way of this definition, so there are some exceptions.

To say that something is \textit{only} a dialect, you'd have to be
talking in terms of some language.  Something can not be a dialect
without being a member of a language.

\section*{11-8}

Women generally do not swear as much, and when they do the swearing
forms are usually more mild (more generally socially acceptable) than
when men swear.  Women tend to use longer, more polite greetings; eg.
``Hi, how are you?'' rather than ``Hey.''  This is a result of the
social pressure on women to be polite and nice, where this pressure
generally is not present for men.

\end{document}

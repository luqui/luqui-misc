\documentclass[12pt]{article}

\usepackage{amsmath}
\usepackage{float}

\floatstyle{boxed}
\newfloat{Figure}{tbp}{}

\title{Syntactic Typology of Programming Languages}
\author{Luke Palmer}

\begin{document}
\maketitle

\textit{\textbf{Abstract:} Programming languages, interfaces for telling
computers what to do designed in a Human-friendly way, exhibit many of
the same phenomena as natural Human languages.  This paper categorizes
programming languages along two orthogonal axes. It then describes some
of the properties that each programming language family has and
similarities between programming language culture and natural language
culture.}

\section{Paradigm Axes}

It is widely believed throughout the programmer's community that there
are four fundamental paradigms of programming: Procedural,
Object-oriented, Functional, and Logical.   However, \cite{Palmer-2005}
provides an analysis enumerating six different paradigms as a
combination of two axes, called the \textit{control style} and the
\textit{abstraction style}.  These are semantic styles and in theory
they have nothing to do with the syntax, but in practice, they are
highly correlated to syntax.

Computer programs are built inductively, by building a complex structure
out of smaller forms of the same structure.  The first axis, control
style, indicates what kind of structure is being built.  It takes on
three values: imperative, functional, and logical.  

\begin{description}
\item[Imperative] programs build a sequence of instructions for the
computer to execute like a cooking recipe.  This is done inductively
using \textit{procedures}, which are smaller sequences of instructions
that are combined together to make the final program.  For example, they
may issue an instruction ``ask for the user's age'', then do some
computation and decide to issue the instruction to ``display `you are
too old'{}''.  Imperative programs are so called because every statement
is read as though it were in the imperative mood.
\item[Functional] programs build a large mapping from input to output,
out of smaller such mappings called \textit{functions}.  As a contrast
to the imperative paradigm, a functional program would implement a
function which takes a number as input (representing the user's age) and
returns a string (the description of their age) as output. 
\item[Logical] programs build a complex boolean (true or false)
condition that the computer tries to make true by assigning to
variables.  The logical languages do not make up an interesting
syntactic category, so this is the last we will hear of them.
\end{description}

The other axis, abstraction style, indicates how the smaller pieces are
layed out to be combined.  There are two common abstraction styles:
procedural and object-oriented.

\begin{description}
\item[Procedural] abstraction is essentially a substitution filter.  For
example, we could build a procedural English sentence like so: Let
$\mathit{L}(x)$ = ``I like $x$''; I told Mark that $L(\text{Jane})$.
\item[Object-oriented] abstraction, instead, defines names as conceptual
entities to be associated to some object, whose behavior may differ
based on the type of that object.  English is already object-oriented in
a sense, where we can say ``I rode my horse'', and ``I rode my bike''.
The speaker is riding something, but the way in which a horse and a bike
are ridden are quite different, even though they have the same
grammatical structure.
\end{description}

Figure \ref{fig:paradigm-table} presents table of common languages, most
of which we will discuss in this paper, categorized by their control and
abstraction styles.  Many of these languages are multi-paradigm, so they
have been categorized by their most common use.

\begin{Figure}
\label{fig:paradigm-table}
\begin{tabular}{l|ll}
                      & \textbf{Procedural}  & \textbf{Object-oriented} \\
  \hline
  \textbf{Imperative} & C, Pascal, ALGOL     & C++, Java, Smalltalk \\
  \textbf{Functional} & Scheme, ML           & O'Caml, Haskell, Lisp \\
  \textbf{Logical}    & Prolog, Curry        & \textit{none} \\
\end{tabular}
\caption{Common languages categorized by their control and abstraction
styles.} 
\end{Figure}

\section{Control Style}

\subsection{Imperative Syntax}

An extreme majority of languages used today are imperative.  According
to \texttt{freshmeat.net} statistics\cite{Welton-2004}, the top six
languages in use in open source projects are C, Java, C++, Perl, PHP,
and Python, which account for 95\% of all projects.  \textit{All six} of
these are imperative languages\footnote{Perl could be considered on the
border between imperative and functional, but common use is in its
imperative form.}.  C, Java, and C++ have very similar syntax to each
other (we will call these C languages), Perl and PHP are syntactic
derivatives of the C languages.  Python is the only outlier, having a
very different ``first glance'' look from the other five, but some
analysis shows that it is still very close.

These six languages are almost precisely the same in terms of syntax.
Their differences could be considered morphological, or even literal.
The comparison chart in Figure \ref{fig:imperative-comparison-chart}
shows some of the differences.

\begin{Figure}
\label{fig:imperative-comparison-chart}
\begin{tabular}{l|l|l|l}
\textbf{C/Java/C++} & \textbf{Python} & \textbf{Perl}     & \textbf{PHP} \\
\hline
\verb+int x = 1;+   & \verb+x = 1+        & \verb+my $x = 1;+   & \verb+$x = 1+     \\
\verb|x++;|         & \verb|x = x + 1|    & \verb|$x++;|        & \verb|$x++;|      \\
\verb+array+		& \verb+array+        & \verb+@array+		& \verb+$array+     \\
\verb+array[4]+		& \verb+array[4]+     & \verb+$array[4]+    & \verb+$array[4]+  \\
\verb+foo(4)+       & \verb+foo(4)+       & \verb+foo 4+        & \verb+foo 4+      \\
\verb+obj.member+   & \verb+obj.member+   & \verb+$obj->member+ & \verb+$obj->member+ \\
\verb+obj.method()+ & \verb+obj.method()+ & \verb+$obj->method+ & \verb+$obj->method()+ \\
\end{tabular}
\caption{Contrast in the C language family.}
\end{Figure}

There seems to be a fair amount of variation, but the word order is
always the same, and the operators, when not identical, can easily be
transliterated.  Also keep in mind that, except for constructs present
in one language and completely missing in another, these are almost the
only the variations.

There are some key differences.  The largest variation is demonstrated
on the first line where the programmer declares a new variable.  Each of
the languages does something different here.  The C-languages declare a
new variable by stating its type (\texttt{int} (for integer) in this
case, but it could easily be \texttt{float}, \texttt{class String},
\texttt{double (* const)(int, char)}, etc.).  None of the other
languages declare a type.  Perl uses the placeholder \texttt{my} to
indicate that a variable is being declared.  Python uses the fact that a
variable is being assigned to declare the variable.  PHP uses any
reference of a variable to be an implicit declaration.  The following
example clarifies the distinction between the last two:

\begin{verbatim}
# Python code
x = 4
def foo():
    x = x + 1  # make x 5

# PHP code
$x = 4
function foo() {
    $x = $x + 1;  # declare a new $x (defaulting to 0)
                  # and add 1 to it
}
\end{verbatim}

Another interesting distinction is that Python uses indentation to
determine where blocks are, whereas all the other languages use curly
braces.  The above example demonstrates that as well.  This is where
Python gets it unique look.

\subsection{Functional Syntax}

Functional languages account for even less than the remaining 5\% of the
most common languages.  Nonetheless, there is much more variety in the
syntax of functional languages than that of the imperative languages.
There are two possible reasons for this: 

\begin{enumerate}
\item All of the functional languages covered here emerged out of
academic research, whereas most of the imperative languages above
emerged out of industrial research.  Academics are more willing to
experiment with strange new ideas, because there is not nearly as much
on the line if they get it wrong.  
\item Imperative languages have a fairly straightforward compositional
approach: do this, then do that.  Complex units are combined in simple
ways.  Functional languages instead combine simple units in complex
ways.  That is, imperative programming is about content words;
functional programming is about function words.
\end{enumerate}

We will focus on two language families: Lisp/Scheme (the Lisp family)
and ML/Haskell/O'Caml (the ML family). Lisp and Scheme have precisely
the same syntax (they differ in their semantic models and libraries).
ML and O'Caml have similar syntax, but O'Caml adds some features.  Figure
\ref{fig:functional-comparison} gives a comparison of some syntactic
features of Lisp, ML, and Haskell.

\begin{Figure}
\label{fig:functional-comparison}
\begin{tabular}{l|l|l}
\textbf{Lisp}            & \textbf{ML}                   & \textbf{Haskell} \\
\hline
\verb+(let ((x 4)) x)+   & \verb+let val x = 4 in x end+ & \verb+let x = 4 in x+ \\
\verb|(+ x 1)|           & \verb|x + 1|                  & \verb|x + 1| \\
\verb|(if 'nil 0 1)|     & \verb+if false then 0 else 1+ & \verb+if False then 0 else 1+ \\
\verb|(lambda (x) (+ x 1))| & \verb|fn x => x + 1|       & \verb|(+1)| \\
\verb|(foo x y)|         & \verb|foo x y|                & \verb|foo x y| \\
\end{tabular}
\caption{Contrast in the functional languages.}
\end{Figure}

It would take more space than we have here to explain all the
differences demonstrated in that table.  One thing to notice is that
Lisp is minimalist in a sense.  All function calls look like
\texttt{(func arg1 arg2 ...)}, and everything is a function.  ML and
Haskell provide a lot more ``syntactic sugar'', to make programs easier
for Humans to read.

\section{Abstraction Syntax}

\begin{Figure}
\label{fig:procedural-comparison}
\begin{tabular}{l|l|l}
\textbf{C}     & \textbf{Scheme} & \textbf{ML} \\
\hline
\verb+func(x)+ & \verb+(func x)+              & \verb+func x+ \\
\verb+&fptr+   & \verb+(lambda (x) (fptr x))+ & \verb+fptr+ \\
\verb+int foo() {...}+ & \verb+(define (foo) (...))+ & \verb+fun foo ():int+ \\
\end{tabular}
\caption{Contrast across procedural languages.}
\end{Figure}

\begin{Figure}
\label{fig:object-oriented-comparison}
\begin{tabular}{l|l|l}
\textbf{C++} & \textbf{Smalltalk} & \textbf{Haskell} \\
\hline
\verb+obj.meth(arg)+ & \verb+obj meth: arg+ & \verb+meth arg obj+ \\
\verb+class Foo : Bar+ & \verb+Bar subclass: #Foo+ & \verb+class (Bar a) => Foo a+ \\
\verb+int foo(int x) {...}+ & \verb+foo | x | ...+ & \verb+foo x self = ...+ \\
\end{tabular}
\caption{Contrast across object-oriented languages.}
\end{Figure}

Figures \ref{fig:procedural-comparison} and
\ref{fig:object-oriented-comparison} compare features across the
procedural and functional languages in different control styles.  There
is vast variation here: word and clause order, naming conventions,
braces, argument passing styles, etc.  In all of these languages,
classes and objects (the units that make up object-oriented programming)
behave approximately the same, but the programmers from different
languages think about these concepts very differently.  

For example, the second line defines \texttt{Foo} as a subtype of
\texttt{Bar} (for example we would define \texttt{Dog} as a subtype of
\texttt{Animal}).  However, C++ programmers would translate into English
as ``Define a class \texttt{Foo} derived from \texttt{Bar}''.  Smalltalk
programmers would translate ``Tell \texttt{Bar} to subclass itself into
\texttt{Foo}''.  Haskell programmers would say ``Any \texttt{Foo} must
also be a \texttt{Bar}''.  The end meaning is the same, but the mental
space when dealing with these concepts is very different.  

\section{Relation to Natural Language}

Constructs in programming languages are usually inspired by constructs
in Human language.  Every language described here has been developed by
native English speakers\footnote{With the exception of C++, created by
Danish computer scientist Bjarne Stroustrup.  But C++ is such a close
derivative of C that it doesn't really count.}.

There is one very interesting languages to look at when considering
natural language.  This is Perl, created by linguist Larry Wall and
modeled after English.

Perl was designed as an ``idiomatic'' language.  That is, Perl gives the
programmer a lot of syntactic freedom, and relies on the culture to
enforce certain styles to maintain readability.  In this way, Perl
gathers a very Human-like culture.  ``Foreign'' speakers---those who
come from other programming language backgrounds---write Perl as if it
were a transliteration of the language they are comfortable with.  The
more a person codes in Perl, the more it starts to look ``native'', but
programmers may always maintain some kind of ``accent''.    Here are
three ways in Perl to compute the factorial of a number \texttt{\$n}.

\begin{description}
\item[Perl with a C accent]
\begin{verbatim}

my $result = 1:
for (my $i = 1; $i <= $n; $i++) {
    $result *= $i;
}
\end{verbatim}

\item[Perl with a functional accent] 
\begin{verbatim}

sub fact {
    my ($n) = @_;
    if ($n == 0) { 1 }
    else         { $n * fact($n-1) }
}
my $result = fact($n);
\end{verbatim}

\item[Idiomatic Perl]
\begin{verbatim}

my $result = reduce { $a * $b } 1..$n;
\end{verbatim}
\end{description}

Another phenomenon to explore in programming languages is the sentential
complexity present in functional programs.  Consider the following
``Schwartzian Transform'' in Haskell:

\begin{verbatim}
map snd (sort (map (\x -> (f x,x)) list))
\end{verbatim}

The two important things in that sentence are \texttt{sort} and
\texttt{list}, both of which are hidden in the depths of the
expression.  Also note the following ``center embedded''
sentence:

\begin{verbatim}
maybe Nothing (maybe Nothing (maybe Nothing (lookup locations)) 
        (lookup addresses)) (lookup names) x
\end{verbatim}

Such expressions are common in functional programs, and notoriously hard
to read.  That is to be expected, as it is equally hard to read the
following example from \cite{Abe-Hatasa-Cowan-1988}: ``The boy the girl
the teacher flunked kissed cried''.  This may be one of the hidden
causes preventing functional programming from making it into mainstream
code culture.

\section{Conclusions}

The control style of a programming language correlates with its
syntactic structure, but the abstraction style shows no correlation.
This could indicate that the control style determines the way people
think a more heavily than the abstraction style.  Categorizations of the
world vary in usefulness across domains, and Humans are good at
understanding new categorizations because they depend on it to survive.
The abstraction style of a programming language is a categorization of
the concepts of a program.  However, Humans are typically not very good
at coming up with new interpretations of how the universe as a whole
behaves (quantum mechanics, mathematically simple at its core, continues
to haunt and perplex even accomplished physicists).  Rethinking how your
program executes---the control style---is analogous to rethinking how
the universe behaves.  This is hard for Humans, so new concepts in that
area are rarely defined, so new language structures don't have any way
to get in.

Even in explicitly-designed, special-purpose programming languages, we
find a lot of the syntactic phenomena found in natural languages leaping
out of anyone's specific intention.  From the Mandarin Lisp programmer
who can not comprehend the Cantonese Haskell programmer to the
Californian C++ programmer who just thinks New York Java is rude,
language boundaries and attitudes form inside programming cultures.  C
programmers speak an awkward Perl accent and seek help from their
experienced, native friends in the culture.  It seems that such
linguistic phenomena appear whenever humans have to learn any
syntax-semantic mapping.

\newpage

\begin{thebibliography}{4in}
\raggedright
\bibitem{Abe-Hatasa-Cowan-1988} Abe Y., Kazumi H., Cowan J R. (1988).
Investigating universals of sentence complexity.  \textit{Typological
Studies in Language 17}, 77--92.
\bibitem{Palmer-2005} Palmer L. (2005).  \textit{What is the fourth
paradigm?}  Retrieved 2005-05-05 from 
\verb+http://use.perl.org/~luqui/journal/27651+.
\bibitem{Randal-Sugalski-Toetsch-2003} Randal A., Sugalski D., T\"otsch
L. (2003).  \textit{Perl 6 Design Philosophy}.  Retrieved 2005-05-05
from \verb+http://www.perl.com/pub/a/2003/06/25/perl6essentials.html+.
\bibitem{Welton-2004} Welton D. (2004). \textit{Programming Language
Popularity}.  Retrieved 2005-05-05 from
\verb+http://www.dedasys.com/articles/language_popularity.html+.
\bibitem{Wikipedia-C-programming-language} Wikipedia.  \textit{C programming
language}.  Retrieved 2005-05-05 from 
\verb+http://en.wikipedia.org/wiki/C_programming_language#History+
\end{thebibliography}

\end{document}

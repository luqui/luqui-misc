\documentclass[12pt]{article}

\usepackage{fancyhdr}
\usepackage{amsmath}

\lhead[\fancyplain]{Luke Palmer}
\chead[\fancyplain]{2006-09-13}
\rhead[\fancyplain]{LING 3430}
\pagestyle{fancyplain}

\begin{document}
\section*{2.1}
\begin{description}
\item[a.] The speaker is referring to Nairobi Airport.
\item[b.] The speaker is not referring; they lack everything in the
class of items known as ``food''.
\item[c.] The speaker is referring to the particular pair of shoes that
fell out of the cupboard.
\item[d.] The speaker is not referring; Henry is going to make something
in the class ``cake''.
\item[e.] The speaker is not referring; Doris passed through the office
like any whirlwind, not a particular one.
\item[f.] The speaker is referring to the particular bus taht ran
``him'' over.
\item[g.] The speaker is not referring; any army of volunteers would do.
\end{description}

\section*{2.2}
\begin{description}
\item[a.] Kiev
\item[b.] Tony Blair
\item[c.] December 25
\item[d.] The United States Capital
\item[e.] Mount Everest or Mauna Kea (depending on your definition of
``tall'')\footnote{I got this info, appropriately, from the START
natural language question answering system from MIT.}
\end{description}

I don't immediately see a problem with referential theory.  The fact
that some of these can depend on when the utterance was made or how you
define some words in the reference just means that you can't be an
absolutist about it.  Like in the rest of this field, you have to
incorporate context.

\section*{2.3}
\begin{description}
\item[a.] Karl Marx was the author of The Communist Manifesto.  Karl
Marx was a German economist.
\item[b.] Alexander Graham Bell invented the Telephone (news to me).
Alexander Graham Bell was a scottish scientist and engineer.
\item[c.] Confucius was a Chinese philosopher.  Confucius was the
founder of the system of philosophy ``Confucianism''.
\item[d.] James Joyce was an Irish writer.  James Joyce wrote Ulysses.
\item[e.] Alexander the Great was a Macedonian military commander.
Alexander the Great conquered a lot of places.
\item[f.] Indira Gandhi was the prime minister of India.  She isn't
related to Mahatma Gandhi.
\end{description}

I have knowledge of these people (except for Bell, Hoyce, and Gandhi,
who I had no knowledge of) through social osmosis.  At least that was
the case with Confucius and Alexander; I have studied economics and the
Russan revolution, so I know about Marx from those.  Alexander the
Great's fable has presumably been around for 2500 years, passed down
through social chains.

\section*{2.4}
I'll use the notation $A \rightarrow B$ to mean both ``B is a necessary
condition for A'' and ``A is a sufficient condition for B''.

Note that while some of these may share the given properties, the
implications are unique.  For example, a biscuit can be hard, but being
a biscuit does not imply being hard, so biscuit and cracker are
distinct.

\begin{description}
\item[a.] \textbf{cake} $\rightarrow$ frosting; \textbf{biscuit}
 $\rightarrow$ unraised; biscuit or roll or bun $\rightarrow$
 \textbf{bread}; \textbf{roll} $\rightarrow$ rounded; \textbf{bun}
 $\leftrightarrow$ roll (synonyms, as far as I can tell);
 \textbf{cracker} $\rightarrow$ hard.
\item[b.] \textbf{boil} $\rightarrow$ water; \textbf{fry} $\rightarrow$
 oil; \textbf{broil} $\leftrightarrow$ broiler (yeah, redundant, but it's
 distinguishing and correct); \textbf{saut\'e} $\rightarrow$ quickly;
 \textbf{simmer} $\rightarrow$ low heat; \textbf{grill} $\leftrightarrow$
 a grill; \textbf{roast} $\rightarrow$ closed oven (disputable with
 broil).
\end{description}

\section*{2.5}

For the following concepts, I kept finding my explanation for why the
atypical example was atypical to be ``if you asked somebody to draw it,
that is very unlike what they would draw''.  The atypical examples lack
many of the attributes listed for the concept, but somehow they are
still in that class (or ``sort of''/``mostly'' in that class).  This
imagery definition seems to be pretty good; if someone says ``mother'',
you start sketching the attributes in your head and refining them as
more information comes in.

\begin{description}
\item[a.] A \textsc{vehicle} moves, has wheels, has a cabin, has a
driver or pilot.  A car is prototypical, a plane a little less so, a
robotically-guided hover board quite unprototypical.
\item[b.] A \textsc{home} has residents, a roof, walls, a yard.  A
prototypical home is my house; an atypical home is a cardboard box.
\item[c.] A \textsc{work} has employees, has cubicles, pays money,
consumes the daytime on weekdays.  A prototypical workplace is
Microsoft (I'm guessing); an atypical workplace is the homeless shelter.
\item[d.] A \textsc{mother} has a son or daughter, makes cookies, nags,
kisses.  A typical mother is the Brady Bunch mom; an atypical mother is
a mother lizard (because it abandons its young and doesn't make
cookies).
\item[e.] A \textsc{science} uses mathematics, causes explosions,
attracts mad geniuses, draws refutable conclusions based on evidence.  A
typical science is Chemistry; an atypical science is Christian Science
(zing), or Philosophy.
\end{description}
\end{document}

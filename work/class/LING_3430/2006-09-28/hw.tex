\documentclass[12pt]{article}

\usepackage{fancyhdr}
\usepackage{amsmath}

\lhead[\fancyplain]{Luke Palmer}
\chead[\fancyplain]{2006-09-28}
\rhead[\fancyplain]{LING 3430}
\pagestyle{fancyplain}

\begin{document}
\section*{3.2}
\begin{description}
\item[(safe/secure)] He plays it safe. \textit{vs.} *He plays it secure.
\item[(quick/fast)] Let's get fast food.  \textit{vs.} *Let's get
quick food.
\item[(near/close)] He had a near-death experience.  \textit{vs.} *He
had a close-(to)-death experience.
\item[(fake/false)] She put on a fake smile.  \textit{vs.} (*?) She put on a
false smile.
\item[(sick/ill)] I'm getting sick.  \textit{vs.} *I'm getting
ill.\footnote{As contrasted to ``I'm falling/becoming ill''.}
\item[(expensive/dear)]  I'm not familiar with this sense of
\textit{dear}.  Let's try anyway:  That was an expensive car.
\textit{vs.} *That was a dear car.
\item[(dangerous/perilous)] That is a dangerous offer. \textit{vs.}
*That is a perilous offer.
\item[(wealthy/rich)] His inheritance made him stinking rich.
\textit{vs.} *His inheritance made him stinking wealthy.
\item[(mad/insane)] It was conjured up by a mad scientist.  \textit{vs.}
*It was conjured up by an insance scientist.
\item[(correct/right)] I apologized to make it right. \textit{vs.} *I
apologized to make it correct.  \footnote{Maybe this invokes the wrong
sense of \textit{right}.}
\end{description}

\section*{3.3}
``Powell complained about the new case, and Lorenz did too.''  Of the
three possibilities, you could not use one interpretation for Powell's
complaints and a different for Lorenz, which suggests that it is
\textit{ambiguous}.

\begin{enumerate}
\item He packed his clothes into his new suitcase.
\item *He packed his clothes into his new investigation.
\item *\footnote{We hope not.}He packed his clothes into his new patient.
\item The new investigation stumped the whole department.
\item The new patient stumped the whole department.
\item *The new suitcase stumped the whole department.
\end{enumerate}

It is \textit{possible} that, on account of (4) and (5), ``case'' is
vague for police case or medical case.  However, the do so identity test
claims that it is a different sense.  A do so identity test that might
possibly succeed would be ``The hospital was stumped on a new case,
and so was the police department.''  I would still consider it strange.

\section*{3.4}
\begin{description}
\item[(temporary/permanent)] Probably simple antonyms, because if (say)
a condition is not temporary, then it is permanent.
\item[(monarch/subject)] Converses, because if Caesar is my monarch,
then I am his subject.
\item[(advance/retreat)] Reverses, because if I advance and then
retreat, I am in the same state as I was to start with.
\item[(strong/weak)] Gradable antonyms, because (1) you can express
relativity as in ``I am stronger than you are.'', and (2) because there
is a default as in ``How strong are you?'' as opposed to ``How weak are
you?''  Interestingly, I don't know of any intermediate terms.
\item[(buyer/seller)] Converses, because if I sold a car to John, then I
am the seller and he is the buyer.
\item[(boot/sandal)] Taxonomic sisters: they are both footwear.
\item[(assemble/disassemble)] Reverses, because if I disassemble
something and then assemble it, it is in the same state as when I
started.
\item[(messy/neat)] Gradable antonyms, because there is relativity: ``My
room is messier than Jude's.''  There are also stronger forms: ``pig''
\textit{vs.} ``anal retentive''.
\item[(tea/coffee)] Taxonomic sisters: they are both drinks.
\item[(clean/dirty)] I think I can argue the same as messy/neat.  In
some situations they are simple antonyms (``is that glass clean?'',
``no, it is dirty''), but there is also relativity.
\item[(open/shut)] Simple antonyms: ``*this door is more open than that
one''.  If a door is not shut, then it is open.
\item[(friend/enemy)] I am going to say it is a simple antonym, because
I've decided that the ``if it's not one, it's the other'' test is bogus,
and replace it with ``it cannot be both''.\footnote{For example ``My
glass is not dead, so it must be alive'' is not logically sound.}  If
you are my friend, you cannot be my enemy, and if you are my enemy, you
cannot be my friend; however you can be neither.
\end{description}

\section*{3.5}
I'll use the notation $\subset$ for ``is a hyponym of.''

Piano $\subset$ keyboard $\subset$ instrument $\subset$ device $\subset$
entity.

Spin-up electron $\subset$ electron $\subset$ particle $\subset$ object
$\subset$ entity.

I'll take this time to discuss the notion that a taxonomy requires a
tree ordering rather than just a partial ordering, which strikes me as a
little strange.  One of the requirements of sisters in a taxonomy is
that they are mutually exclusive.  This ends up entailng that any two
classes anywhere on different branches are disjoint, i.e. that a
taxonomy is a tree.  But there are terms which don't satisfy this (e.g.
all mammals are animals and all underwater creatures are animals, but
whales are both), which implies that such terms cannot be in the same
taxonomy.

So let's make a supertaxonomy out of all lexical taxonomies, where $a
\subset b$ in the supertaxonomy if and only if $a \subset b$ in some
lexical taxonomy.  It's easy to show that this is still transitive, and
it allows statements like the above about mammals.  Why not use this as
the taxonomy instead?  What does requiring the tree ordering get you?
Sorry about the math \texttt{:-)}.

\end{document}

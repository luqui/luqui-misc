\documentclass[12pt]{article}

\usepackage{fancyhdr}
\usepackage{amsmath}

\lhead[\fancyplain]{Luke Palmer}
\chead[\fancyplain]{2006-10-05}
\rhead[\fancyplain]{LING 3430}
\pagestyle{fancyplain}

\newcommand{\true}{\mathbf{T}}
\newcommand{\false}{\mathbf{F}}
\newcommand{\reduc}{\;\vdash\;}

\begin{document}
\section*{4.3}
(I'll use $\vdash$ to indicate reduction steps)
\begin{description}
\item[S1] 
 \begin{gather*}
   \neg (p \vee q) \reduc \neg (\true \vee \false)
    \reduc \neg \true \reduc \false \\
   \neg p \vee q \reduc \neg \true \vee \false
    \reduc \false \vee \false \reduc \false
 \end{gather*}
\item[S2]
 \begin{gather*}
  \neg (p \vee q) \reduc \neg (\true \vee \true)
   \reduc \neg \true \reduc \false \\
  \neg p \vee q \reduc \neg \true \vee \true
   \reduc \false \vee \true \reduc \true
 \end{gather*}
\item[S3]
 \begin{gather*}
  \neg (p \vee q) \reduc \neg (\false \vee \true)
   \reduc \neg \true \reduc \false \\
  \neg p \vee q \reduc \neg \false \vee \true
   \reduc \true \vee \true \reduc \true
 \end{gather*}
\end{description}

\section*{4.5}
\begin{description}
\item[a.] ``The train will either arrive or it won't arrive.'' This is
not analytically true; it is a logical tautology (of the form $p \vee
\neg p$).
\item[b.] ``If it rains, we'll get wet.''  I think this has to be
synthetic, because the speaker could be holding an umbrella, which would
make the statement false.
\item[c.] ``Every doctor is a doctor.''  Though Saeed's discussion
classified ``my father is my father'' as analytic, I tend to disagree.
I'd say this is a tautology.  Denote ``$x$ is a doctor'' by $p(x)$.
Then this statement is of the form $\forall x. p(x) \Rightarrow p(x)$,
which is a logical tautology; i.e. it holds no matter what $p$ means.
If I understood the discussion correctly, analytic truths are
tautological only once you incorporate the meanings of the words.
\item[d.] ``If albert killed a deer, then Albert killed an animal.''
Finally, this is analytic, because in our language, a deer is an animal,
so this statement must be true (but only after knowing the relationship
between deer and animal).
\item[e.] ``Madrid is the captial of Spain.'' This is synthetic, as it
is concievable that Madrid could, one day, not be the capital of Spain
anymore.
\item[f.] ``Every city has pollution problems.''  This is synthetic, as
it may not even be true right now.
\end{description}

\section*{4.6}
\begin{description}
\item[1.] $a$ entails $b$, because ``passed'' and ``failed'' are simple
antonyms (i.e. ``passed'' is the same as ``didn't fail'').
\item[2.] $a$ entails $b$, because, well, of the relationship between
the words ``inherit'' and ``own''.
\item[3.] $a$ does not entail $b$, since Cassidy could have sold the
farm.
\item[4.] $a$ entails $b$ (under the assumption that the word ``poison''
implies ``kill'', rather than just ``make sick'' or something).
\item[5.] $a$ entails $b$, since $b$ is just the passive construction of
$a$ (thus $b$ entails $a$ also).
\item[6.] $a$ does not entail $b$, because if nobody liked the show,
then it is still true that ``not everyone'' liked the show.
\end{description}

Um, how were we supposed to use the composite truth table for those?

\section*{4.7}
\begin{description}
\item[announce] Factive: ``he announced that \#4 took the lead'' and ``he
didn't announce that \#4 took the lead'' both presuppose that ``\#4 took
the lead'' (this is arguable, because just because something is
announced does not make it true).
\item[assume] Not factive: ``she assumed that armageddon was not
coming'' does not presuppose that ``armageddon was not coming''.
\item[be] Not factive.
\item[aware] Factive:  ``he was aware that she had been crying'' and
``he was not aware that she had been crying'' both presuppose that ``she
had been crying''.
\item[believe] Not factive.  Neither ``John believes that aliens exist''
nor ``John doesn't believe that aliens exist'' presuppose that ``aliens
exist''.
\item[be fearful] Hmmm, not factive, I think.  ``Sue was fearful of
aliens coming'' does not presuppose that aliens were coming.
\item[be glad] Factive.  ``Jamie was glad that it was christmas'' and
``Jamie was not glad that it was christmas'' both presuppose that it was
christmas.
\item[be sorry] Factive.  Use the same sentence as above.\footnote{Yeah,
I'm lazy, I know.}
\item[be worried] Not factive.  ``Paul was worried that the red coats
were coming'' does not presuppose that they were coming.
\item[know] Factive.  ``Daniel knew that there was a blue horse'' and
``Daniel didn't know that there was a blue horse'' both presuppose that
``there was a blue horse''.
\item[reason] Factive.  That is arguable, because ``Willard Quine
reasoned that `this statement is false' must be false'' doesn't
presuppose that what he reasoned was true, only that he reasoned it.
\item[reported] Factive.  ``Jimmy reported that the ant colony had
grown'' and ``Jimmy did not report that the ant colony had grown'' both
presuppose that ``the ant colony had grown''.
\end{description}

\section*{4.8}
The following will be the reasoning, so I don't have to monotonously
repeat it.  If I say ``presuppose'', then I negated the sentence and $b$
was still true.  If I say ``entail'', then I negated the sentence and
$b$ did not necessarily follow anymore.

\begin{description}
\item[1.] Presuppose.
\item[2.] Entail.
\item[3.] Presuppose.
\item[4.] Entail.
\item[5.] Presuppose.
\end{description}
\end{document}

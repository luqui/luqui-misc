\documentclass[12pt]{article}

\usepackage{longtable}
\usepackage{fancyhdr}
\usepackage{amsmath}

\lhead[\fancyplain]{Luke Palmer}
\chead[\fancyplain]{2006-10-05}
\rhead[\fancyplain]{LING 3430}
\pagestyle{fancyplain}

\begin{document}
\section*{8.3}
\begin{longtable}{l||l|l}
& \textbf{Direct} & \textbf{Indirect} \\
  \hline
a. & question & offer \\
b. & statement & request \\
c. & statement\footnote{I'm not totally clear on this column.  Should
this one be ``promise'', since the speaker is ``promising to bet''
something (i.e. using future tense)?  Or is statement sufficient?} &
question / conclusion / offer \\
d. & statment & denial (a \textit{declaration}) \\
e. & command & some \textit{expressive}\footnote{Contempt? No, that's
not the right word.  My mental thesaurus is broken.} \\
f. & command & denial \\
\end{longtable}

\section*{8.4}
\begin{description}
\item[a.] You know, you could work on some homework.
\item[b.] Can you take a picture of that landscape?
\item[c.] I'd like you to keep quiet for a moment.
\item[d.] Will you take me to practice tomorrow?
\end{description}

\section*{8.5}
\subsection*{Promising}
\begin{description}
\item[a.] *Cannot seem to make something work here*
\item[b.] [After being asked: ``Could you take that movie back?''] You
didn't know I was going to take it back on my way to work tomorrow?
\item[c.] I'm going to buy you some tea today.
\item[d.] Have you noticied that I've been falling behind on my
homework, which I intend to stop doing?  (*This feels like cheating*)
\end{description}

Both (a.) (stating the preparatory condition) and (d.) (querying the
propositional content) in order to make a promise seemed very difficult.

\subsection*{Questioning}
\begin{description}
\item[a.] I don't know what the name was of that guy who visited last
week.
\item[b.] Are you going to tell me where you want to eat?
\item[c.] I'd like to know whether you love me.
\item[d.] What is that animal called?
\end{description}

Assuming I interpreted (d.) correctly, they all went pretty smoothly.
(a.) seems a little awkward, but I'm not convinced that stating the
preparatory condition for questions will always be awkward.

\end{document}

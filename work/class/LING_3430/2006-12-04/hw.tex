\documentclass[12pt]{article}

\usepackage{longtable}
\usepackage{fancyhdr}
\usepackage{amsmath}

\lhead[\fancyplain]{Luke Palmer}
\chead[\fancyplain]{2006-12-04}
\rhead[\fancyplain]{LING 3430}
\pagestyle{fancyplain}

\begin{document}
\section*{9.2}
\begin{description}
\item[mail] Class 1b.  Sally mailed a dog to her son.  Sally mailed her
son a dog.
\item[push] Class 2.  I pushed the books to John.  (?)I pushed John the
books.  The latter sentence is a little strange, so I think the
prediction still works.
\item[kick] Class 1b.  Fernando kicked the ball to Julia.  Fernando
kicked Julia the ball.
\item[pass] Class 1b.  Fernando passed the ball to Julia.  Fernando
passed Julia the ball.
\item[sell] Class 1a.  Zaphod sold the drive to Arthur.  Zaphod sold
Arthur the drive.
\item[lower] Class 2.  Paul lowered another girder to Rodriguez.  (?)Paul
lowered Rodriguez another girder.
\item[hand] Class 1a.  Le\'o handed the letter to Franklin.  Le\'o
handed Franklin the letter.
\item[flip] I don't see how this is a transfer-of-possession verb.
\item[throw] Class 1b.  Fernando threw the ball to Julia.  Fernando
threw Julia the ball.
\item[bring] Class 1a.  Sally brought a dog to her son.  Sally brought
her dog son a dog.
\item[haul] Class 2.  Tennesse hauled coal to his boss.  (?)Tennesse hauled
his boss coal.
\item[ferry] Class 1b.  Albert will ferry the letter to Franklin.
Albert will ferry Franklin the letter.
\item[take] Class 1a.  Sally took a dog to her son.  Sally took her son
a dog.
\end{description}

The prediction seems to be accurate, but it is hard to tell with
certainty given so few class 2 examples.  Also, the dative forms of the
class 2 examples sounded usually just a little awkward, not outright
wrong.

\section*{9.3}
\begin{description}
\item[teach] Class 3: (?)Steve taught the answer to Luke.  Steve taught
Luke the answer.  This is an interesting case, since the prepositional
case sounds wrong.
\item[read] Class 4: Rob read a book to Cory.  Rob read Cory a book.
Both of these seem valid to me, contradicting the prediction.
\item[whisper] Class 4: Jenny whispered her secret to Carol.  (?)Jenny
whispered Carol her secret.
\item[mention] Class 4: Steve mentioned his radioactivity to his
audience.  *Steve mentioned his audience his radioactivity.
\item[quote] Class 4: Mike quoted J.P. to the student.  *Mike quoted the
student J.P.
\item[murmur] Class 4: Luke murmured something to Sam.  *Luke murmured
Sam something.
\item[say] Class 4: Nolan said he wanted to play again to Josh.  *Nolan
said Josh he wanted to play again.  I wouldn't consider ``say'' in
either of the two classes, considering that its argument has to be a
quote or a summary of a quote, and can't just mention something.
\item[show] Class 3: (?)Rick showed how to prove it to me.  Rick showed
me how to prove it.  Again, the prepositional case sounds weird.
\item[scream,yell] Class 4: These two behave the same as ``shout'', given in the
example.
\item[cite] I don't see how this is a communication verb in the sense
they are describing; i.e. I don't see how you could cite something
``to'' someone.
\end{description}

The prediction seems almost accurate.  It looks like class 3 verbs have
a tendency to only work with the dative form, whereas class 4 verbs only
work with the prepositional form.  ``read'' is a counterexample, but
read has properties of both classes (for example, you can ``read someone
to sleep'', so they don't cognitively possess the content).

\section*{9.6}

(b) ``The wind howled through the trees'' does not seem, on the surface,
to fit Croft's model.  However, wind cannot howl without moving (wind
can't even exist without moving), so it does work.

(d) ``The ball thudded into his chest'' works because that implies that
the ball was moving quickly in order to make the thud sound.

All the rest fairly straightforwardly fit into Croft's generalization.

\section*{9.9}

\begin{description}
\item[1a] The window is closed $\Rightarrow$ \\\textsc{$[_{state}$ be $[_{thing}$
window $] [_{place}$ at $[_{property}$ closed $]]]$}.
\item[1b] The window closed $\Rightarrow$ \\\textsc{$[_{event}$ inch
$[_{state}$ be $[_{thing}$ window $] [_{place}$ at $[_{property}$ closed
$]]]]$}.
\item[2a] Peg became angry $\Rightarrow$ \\\textsc{$[_{event}$ inch
$[_{state}$ be $[_{thing}$ Peg $] [_{place}$ at $[_{property}$
angry$]]]]$}.
\item[2b] Bob angred Peg $\Rightarrow$ \\\textsc{$[_{event}$ cause
$[_{thing}$ Bob $] [_{event} $ inch $[_{state}$ be $[_{thing}$ Peg $] [_{place}$ at
$[_{property}$ angry $]]]]]$}.
\item[3a] George had the money $\Rightarrow$ \\\textsc{$[_{state}$ have
$[_{thing}$ George $] [_{thing} $ money $ ]]$}.
\item[3b] George gave the money to Cindy $\Rightarrow$
\\\textsc{$[_{event}$ cause $[_{thing}$ George $] [_{event}$ inch
$[_{state}$ have $[_{thing}$ Cindy $] [_{thing}$ money $]]]]$}.
\item[4a] The prisoners walked into the yard $\Rightarrow$
\\\textsc{$[_{event}$ go $[_{thing}$ prisoners $] [_{place}$ yard $]]$}.
\item[4b] The guards walked the prisoners into the yard $\Rightarrow$
\\\textsc{$[_{event}$ cause $[_{thing}$ guards $] [_{event}$ go
$[_{thing}$ prisoners $] [_{place}$ yard $]]]$}.
\end{description}

\end{document}

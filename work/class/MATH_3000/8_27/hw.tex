\documentclass[12pt]{article}

\usepackage{amsmath}

\begin{document}
\noindent Luke Palmer \\
8/27/2004 \\
MATH 3000

\section*{Page 8}
\begin{description}
\item[2d] 
  \begin{tabular}{lll|l}
    $P$ & $Q$ & $R$ & $(P \wedge Q) \vee (P \wedge R)$ \\
    \hline
     F  &  F  &  F  &  F \\
     F  &  F  &  T  &  F \\
     F  &  T  &  F  &  F \\
     F  &  T  &  T  &  F \\
     T  &  F  &  F  &  F \\
     T  &  F  &  T  &  T \\
     T  &  T  &  F  &  T \\
     T  &  T  &  T  &  T
  \end{tabular}

\item[3d] Not equivalent.  Counterexample: P false and Q false.

\item[4d] True.

\item[4h] False.

\item[4j] True.  Equivalent to just $P$.

\item[5b] We will win neither the first nor the second game.

\item[5h] Sue will choose ice cream or she will not choose yogurt.

\item[7a] $(A \vee B) \wedge \neg (A \wedge B)$  

\item[7b] $\neg (C \vee A)$

\item[7d] $A \wedge \neg C$

\item[11a] This is neither.  In fact, this is equivalent to the equivalence
operator $P \Leftrightarrow Q$.  It is not a tautology, as shown by the
assignment P false Q true, and it is not a contradiction, as shown by the
assignment P true Q true.

\item[11b] This is a tautology, since $P \wedge \neg P$ is obviously a
contradiction.

\item[11d] This is a tautology.  It is a connected series of disjunctions, each
of which covers each possible assignment of P and Q.
\end{description}

\section*{Page 16}

\begin{description}

\item[4b] False.

\item[4f] True (regardless of the fact that I don't know when Euclid was born).

\item[4h] False.

\item[5d] False.

\item[5e] True.

\item[6g] 
  \begin{tabular}{lll|l}
    $P$ & $Q$ & $R$ & $(P \Rightarrow Q) \Rightarrow (R \vee \neg P)$ \\
    \hline
     F  &  F  &  F  &  F  \\
     F  &  F  &  T  &  T  \\
     F  &  T  &  F  &  F  \\
     F  &  T  &  T  &  T  \\
     T  &  F  &  F  &  T  \\
     T  &  F  &  T  &  T  \\
     T  &  T  &  F  &  F  \\
     T  &  T  &  T  &  F
  \end{tabular}

\item[8b] $(n \text{ is prime}) \Rightarrow ((n = 2) \vee (n \text{ is odd}))$

\item[8f] $(S \text{ is compact}) \Leftrightarrow 
           ((S \text{ is closed}) \wedge (S \text{ is bounded}))$

\item[8g] $(\mathbf{B} \text{ is invertible}) \Leftrightarrow
           (\mathit{det} \mathbf{B} \ne 0)$

\item[9d] The only way the former statement can be false is if P is true and
          Q and R are both false.  This follows from the definition of
          implication.  The same holds for the latter statement, as Q needs to
          be false, and P and $\neg R$ both need to be true, requiring that 
          R need be false.  The two statements therefore produce the same
          outcome for all assignments of variables (true for most except this 
          specific case), and are thus equivalent.

\item[14b] Tautology.

\item[14f] Neither (Q true, P false to make it false; Q false to make it true).

\item[14m] Tautology.

\item[15i] $M \Rightarrow (C \wedge \neg A)$ where (M: Fiorello will go to the 
           movies; C: a comedy is playing; A: there \textit{are} 
           advertizements at the beginning).

\item[15ii] $\neg C \wedge ((T \wedge \neg N) \Rightarrow R)$ where (C: Mary 
           will do the crossword; T: delivered on time; N: there are Cal Thomas
           columns; R: she'll read the rest of it).

\end{description}

\end{document}

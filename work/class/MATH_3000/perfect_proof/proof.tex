\documentclass[12pt]{article}
\usepackage[black-square,define-standard-theorems,dont-number-theorems]{QED}
\usepackage{amsmath}
\usepackage{amssymb}

\newcommand{\Dom}{\operatorname{Dom}}
\newcommand{\Rng}{\operatorname{Rng}}

\begin{document}

\begin{Theorem}
 For all functions $f: A \mapsto B,\; g,h: B \mapsto C$, $f$ is onto if
  and only if $g \circ f = h \circ f \; \Rightarrow \; g = h$.
\end{Theorem}

\medskip

\begin{Proof}
\begin{description}
\item[Part 1]  Given $f$ is onto, show $g \circ f = h \circ f$ 
  implies $g = h$.

Assume $g \circ f = h \circ f$.  Since $f$ is onto, for every $y \in B$
there is an $x \in A$ such that $f(x) = y$.  $g(f(x)) = h(f(x))$, so
$g(y) = h(y)$ and $g = h$.

\item[Part 2]  Given $g \circ f = h \circ f$ implies $g = h$, show $f$
  is onto.

% XXX: This is more naturally done by contraposition.
Suppose for the sake of contradiction that $g \circ f = h \circ f$
implies $g = h$ and $f$ is not onto.  Then there exists $y \in B$ but $y
\notin \Rng(f)$.  Let $g$ and $h$ be any functions such that $g
\upharpoonright \Rng(f) = h \upharpoonright \Rng(f)$ and $g(y) \ne
h(y)$.  Then $g \circ f = h \circ f$ but $g \ne h$, contradicting our
assumption. \qed
\end{description}
\end{Proof}

\end{document}

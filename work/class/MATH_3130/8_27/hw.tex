\documentclass[12pt]{article}

\usepackage{amsmath}

\newcommand{\mm}[1]{\begin{pmatrix}#1\end{pmatrix}}
\newcommand{\vv}[1]{\begin{bmatrix}#1\end{bmatrix}}

\begin{document}
\noindent Luke Palmer \\
8/27/2004 \\ 
MATH 3130

\begin{description}
\item[5.]
  \begin{alignat*}
    6&x_1 - 3&x_2 &= 1 \\
   -1&x_1 + 4&x_2 &= -7 \\
    5&x_1    &    &= -5
  \end{alignat*}

\item[6.]
  \begin{align*}
     -2 x_1 + 8 x_2 + 1 x_3 &= 0 \\
     3 x_1 + 5 x_2 - 6 x_3 &= 0
  \end{align*}


\item[9.] \[
  x_1 \vv{0 \\ 4 \\ -1} +
  x_2 \vv{1 \\ 6 \\ 3} +
  x_3 \vv{5 \\-1 \\ -8} =
      \vv{0 \\ 0 \\ 0}
\]

\item[10.] \[
  x_1 \vv{4 \\ 1 \\ 8} +
  x_2 \vv{1 \\ -7 \\ 6} +
  x_3 \vv{3 \\ -2 \\ -5} =
      \vv{9 \\ 2 \\ 15}
\]

\item[11.] \[
  \mathbf{b} = 2\mathbf{a}_1 + 3\mathbf{a}_2
\]

\item[14.] \[
  \mathbf{b} = A \vv{245/33 \\ -41/33 \\ -6/33}
\]

\item[26.] 
  \begin{description}
    \item[a:] Yes: \[
      \mathbf{b} = A \vv{2 \\ 0 \\ 1}
    \]
    \item[b:] The third column of $A$ is simply the linear combination
    $\vv{0\\0\\1}$. 
  \end{description}

\end{description}
\end{document}

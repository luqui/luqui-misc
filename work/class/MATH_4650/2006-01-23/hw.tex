\documentclass[12pt]{article}

\begin{document}
\noindent 
Luke Palmer \\
2005-01-27 \\
Homework 1

\section*{\S1.1}

\begin{description}
\item[2b.] Let $f(x) = 4x^2 - e^x$, and observe that $f \in C[0,1]$.
$f(0) = -1 < 0 < f(1) = 4-e$, so by the intermediate value theorem,
there exists a $c \in (0,1)$ such that $f(c) = 0$.

\item[3d.] Let $f(x) = (x-2) \sin{x} \ln{x+2}$, which is differentiable
on $[-1,3]$.  $f(-1) = 0$ and $f(2) = 0$, so by Rolle's theorem, there
exists a $c \in (-1,2)$ such that $f'(c) = 0$.

\item[7.] Let $f(x) = x^3$, which is infinitely differentiable
everywhere.
  \begin{description}
  \item[a.] By taylor's theorem, $f(x) = F_2(x) + R_2(x)$ where
            $F_2(x) = 0 + 0x + 0x^2/2 = 0$ and 
            $R_2(x) = \frac{f^{(3)}(\xi(x))}{6} x^3 = x^3$.  In this
            case, this works for all values of $\xi$.
  \item[b.] $P_2(0.5) = 0$; $R_2(0.5) = 0.125$.  So $f(0.5) = 0$ with an
            error of approximately exactly $0.125$.
  \item[c.] $f(x) = F_2(x) + R_2(x)$ where $F_2(x) = 1 + 3(x-1) + 6(x-1)^2/2$
            and $R_2(x) = (x-1)^3$.
  \item[d.] $P_2(0.5) = 0.25$; $R_2(0.5) = -0.125$.  So $f(0.5) = 0.25$
            with an error of $0.125$.
  \end{description}
\end{description}

\end{document}

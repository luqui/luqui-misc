\documentclass[12pt]{article}

\usepackage{amsmath}

\begin{document}
\noindent
Luke Palmer \\
MATH 4650 \\
2006-02-10

Note: The problems that make use of computer output either used GHCi or
one of the programs at the end of this homework.

\begin{description}
\item[2.1.4]

\begin{verbatim}
f($x) = $x**4 - 2*$x**3 - 4*$x**2 + 4*$x + 4
Interval = [-2.0, -1.0]
  [-1.0, -2.0]
  [-1.0, -1.5]
  [-1.25, -1.5]
  [-1.375, -1.5]
  [-1.375, -1.4375]
  [-1.40625, -1.4375]
  [-1.40625, -1.421875]
-1.41796875
Interval = [0.0, 2.0]
  [2.0, 0.0]
  [2.0, 1]
  [1.5, 1]
  [1.5, 1.25]
  [1.5, 1.375]
  [1.4375, 1.375]
  [1.4375, 1.40625]
  [1.421875, 1.40625]
1.41796875
Interval = [2.0, 3.0]
  [2.0, 3.0]
  [2.5, 3.0]
  [2.5, 2.75]
  [2.625, 2.75]
  [2.6875, 2.75]
  [2.71875, 2.75]
  [2.71875, 2.734375]
2.73046875
Interval = [-1.0, 0.0]
  [-1.0, 0.0]
  [-1.0, -0.5]
  [-0.75, -0.5]
  [-0.75, -0.625]
  [-0.75, -0.6875]
  [-0.75, -0.71875]
  [-0.734375, -0.71875]
-0.73046875
\end{verbatim}


\item[2.1.12]

\begin{verbatim}
f($x) = $x**2 - 3
Interval = [0.0,3.0] (4)
  [0.0, 3.0]
  [1.5, 3.0]
  [1.5, 2.25]
  [1.5, 1.875]
  [1.6875, 1.875]
  [1.6875, 1.78125]
  [1.6875, 1.734375]
  [1.7109375, 1.734375]
  [1.72265625, 1.734375]
  [1.728515625, 1.734375]
  [1.7314453125, 1.734375]
  [1.7314453125, 1.73291015625]
  [1.7314453125, 1.732177734375]
  [1.7318115234375, 1.732177734375]
  [1.73199462890625, 1.732177734375]
1.73204040527344
\end{verbatim}


\item[2.2.2]

\begin{verbatim}
Prelude> take 5 $ iterate (\x -> (3 + x - 2 * x ** 2) ** (0.25)) 1
[1.0,1.189207115002721,1.0800577526675623,1.1496714305893827,1.1078205295102599]

Prelude> take 5 $ iterate (\x -> ((x + 3 - x**4) / 2**0.5)) 1
[1.0,2.1213203435596424,-10.69759197546794,-9265.829637696228,-5.21221409147882e15]

Prelude> take 5 $ iterate (\x -> ((x + 3)/(x**2+2))**0.5) 1
[1.0,1.1547005383792515,1.116427409872122,1.1260522330022757,1.1236388847132548]

Prelude> take 5 $ iterate (\x -> (3*x**4 + 2*x**2 + 3)/(4*x**3 + 4*x - 1)) 1.0
[1.0,1.1428571428571428,1.1244816900178953,1.1241231639401488,1.124123029704334]
\end{verbatim}


\item[2.2.8]

$g(x) = 2^{-x}$ has a fixed point because it is continuous and $g([1/3,1]) =
[1/\sqrt[3]{2}, 1/2] \subseteq [1/3,1]$.  In addition, $|g(x)| < 1$ for all $x$
in that interval, so the fixed point is unique.

\begin{verbatim}
Prelude> mapM_ print $ take 15 $ iterate (\x -> 2**(-x)) 1.0
1.0
0.5
0.7071067811865476
0.6125473265360659
0.6540408600420695
0.6354978458133738
0.6437186417228692
0.6400610211772396
0.6416858070429984
0.6409635371779632
0.6412845090665851
0.6411418514717377
0.6412052524498624
0.6411770745288387
0.6411895977668723
\end{verbatim}


\item[2.2.15]

Solving $2 \sin(\pi x) + x = 0$ is solving $f(x) = x$ when $f(x) = -2
\sin(\pi x)$.  However, it does not converge.  I tested it to 1,000
iterations.  For the trees' sake, here are the first 15 only.

\begin{verbatim}
Prelude> mapM_ print $ take 15 $ iterate (\x -> -2 * sin (pi * x)) 1.0
1.0
-2.4492127076447545e-16
1.5388857298831052e-15
-9.66910420742986e-15
6.075277348971162e-14
-3.817209337609655e-13
2.3984233624497707e-12
-1.506973843134066e-11
9.468595909483918e-11
-5.949294269809009e-10
3.738051834415167e-9
-2.3486872363473077e-8
1.4757237134577618e-7
-9.272245553853974e-7
5.8259237028454145e-6
\end{verbatim}


\item[2.2.20]

\begin{description}
\item{(a)} Let $f(x) = \frac{1}{2}x + \frac{A}{2x}$.  Notice that
$f(\sqrt{A}) = \sqrt{A}$.  Also notice that $0 < f(x) < x$ whenever $x >
0$.  Therefore $f((0,A)) \subseteq f((0,A))$ and there exists a fixed
point (but we knew that already).  Also, $|f'(x) = \frac{1}{2} -
\frac{A}{2x^2}| < 1$ whenever $x > \sqrt{\frac{A}{3}}$, so that fixed
point is unique and we'll converge to it.  When $0 < x <
\sqrt{\frac{A}{3}}$, I'm not sure how to show convergence.
\item{(b)} Then $f(x) < x$, and it diverges to negative infinity.
\end{description}


\item[2.2.21]

Restate the Lipschitz condition to be: for every $x,y \in [a,b]$ with $x
\not= y$, $\frac{f(x)-f(y)}{x-y} < L$.  We can see that this implies
$f'(x) < L$ for every $x \in [a,b]$ by the definition of the derivative.
That is, the Lipschitz condition is stronger than the assumption made by
the theorem, so the theorem is still valid.


\item[2.2.22]

The statement of this theorem is almost tautological. If $g$ is
continuously differentiable on $(c,d)$, $p \in (c,d)$, $g(p) = p$ and
$|g'(p)| < 1$, then there is some neighborhood $(a,b)$ around $p$ where
the absolute derivative is also less than 1.  Since this is true, $f(a)
> a$ and $f(b) < b$, so $f((a,b)) \subseteq (a,b)$.  Then this
neighborhood satisfies the conditions of the fixed point theorem.


\item[2.3.6e]

\begin{verbatim}
f($x) = exp($x) - 3*$x**2
f'($x) = exp($x) - 6*$x
Initial guess = 0.5
  0.5
  1.16508948243844
  0.936226937560653
  0.910396664872018
  0.910007661863127
  0.910007572488714
0.910007572488709
Initial guess = 4
  4
  3.78436114516737
  3.73537937507954
  3.7330838978741
  3.73307902865468
  3.73307902863281
3.73307902863281
\end{verbatim}


\item[2.3.8e]   

\begin{verbatim}
f($x) = exp($x) - 3*$x**2
f'($x) = exp($x) - 6*$x
Initial guess = secant 0
Initial guess 2 = 0.5
  0, 0.5
  0.5, 4.93687078675298
  4.93687078675298, 0.438951196275084
  0.438951196275084, 0.371867885050184
  0.371867885050184, 1.48270060759859
  1.48270060759859, 0.728462521279268
  0.728462521279268, 0.864018097840986
  0.864018097840986, 0.916037072232311
  0.916037072232311, 0.909838202019623
  0.909838202019623, 0.910006972136034
  0.910006972136034, 0.910007572548784
  0.910007572548784, 0.910007572488709
0.910007572488709
\end{verbatim}


\item[2.3.15]   

The tangent line to $f(x_i)$ is $y(x) = f'(x_i) (x - x_i) + f(x_i)$.
The x-intercept of this function is where $y(x) = 0$. 

\begin{align*}
f'(x_i) (x - x_i) + f(x_i) &= 0 \\
f'(x_i) x - f'(x_i) x_i + f(x_i) &= 0 \\
x &= \frac{f'(x_i) x_i - f(x_i)}{f'(x_i)} \\
x &= x_i - \frac{f(x_i)}{f'(x_i)}
\end{align*}

And Voila!, the canonical presentation of Newton's method.

\end{description}

\section*{Programs Used}

\subsection*{Newton's Method}
\begin{verbatim}
#!/usr/bin/perl

use strict;
use Term::ReadLine;

sub checkzero {
    my ($f, $p) = @_;
    # doesn't work if df/dx > 100 at p
    warn "Warning: Did not converge to a zero!\n"
        unless abs($f->($p)) < 1e-10;
    return $p;
}

sub checkconv {
    my ($a, $b) = @_;
    return abs(($a-$b)/$a) < 1e-12;
}

sub newton {
    my ($f, $df, $p0) = @_;
    my $p = $p0;
    my $prev;

    while (1) {
        print "  $p\n";
        $prev = $p;
        $p -= $f->($p) / $df->($p);

        last if checkconv($p, $prev);
    }
    return checkzero($f, $p);
}

sub secant {
    # basically straight from the book
    my ($f, $p0, $p1) = @_;
    my ($q0, $q1) = ($f->($p0), $f->($p1));
    my $p;

    while (1) {
        print "  $p0, $p1\n";
        $p = $p1 - $q1 * ($p1 - $p0) / ($q1 - $q0);
        last if checkconv($p, $p1);
        ($p0, $q0, $p1, $q1) = ($p1, $q1, $p, $f->($p));
    }
    return checkzero($f, $p);
}

my $pi = 4*atan2(1,1);

my $term = Term::ReadLine::Gnu->new;

my $fntxt = $term->readline('f($x) = ');
my $fn = eval "sub { my \$x = shift; $fntxt }"
            or die "Failed parse: $@";
my $fptxt = $term->readline('f\'($x) = ');
my $fp = eval "sub { my \$x = shift; $fptxt }"
            or die "Failed parse: $@";

while (my $intxt = $term->readline('Initial guess = ')) {
    if ($intxt =~ s/^secant\s*//) {
        my $intxt2 = $term->readline('Initial guess 2 = ');
        print secant($fn, $intxt, $intxt2), "\n";
    }
    else {
        print newton($fn, $fp, $intxt), "\n";
    }
}
\end{verbatim}

\subsection*{Bisection Method}
\begin{verbatim}
#!/usr/bin/perl

use strict;
use Term::ReadLine;

sub bisect {
    my ($func, $lo, $hi, $accuracy) = @_;
    die "Bounds do not satisfy constrants"
        unless $func->($lo) * $func->($hi) < 0;

    if ($func->($lo) > 0) {
        ($lo, $hi) = ($hi, $lo);   # make sure $func->($lo) < 0
    }

    while (abs($hi - $lo) > $accuracy) {
        print "  [$lo, $hi]\n";
        my $mid = ($hi + $lo) / 2;
        if ($func->($mid) < 0) {
            $lo = $mid;
        }
        elsif ($func->($mid) > 0) {
            $hi = $mid;
        }
        else {   # $func->($mid) == 0
            return $mid;
        }
    }
    return(($hi + $lo) / 2);
}

my $term = Term::ReadLine::Gnu->new;

my $fntxt = $term->readline('f($x) = ');
my $fn = eval "sub { my \$x = shift; $fntxt }"
            or die "Failed parse: $@";

while (my $intxt = $term->readline('Interval = ')) {
    unless ($intxt =~ /^ \s* \[
                        \s* (-?\d+\.\d+) \s* ,   # lower bound
                        \s* (-?\d+\.\d+) \s*     # upper bound
                      \] \s* (?:\((\d+)\))? \s*  # optional accuracy
                      $/x) {
        print "Bad interval\n";  next;
    }
    my $accuracy = $3 || 2;
    print bisect($fn, $1, $2, 10**-$accuracy), "\n";
}
\end{verbatim}

\end{document}

\documentclass[12pt]{article}

\usepackage{amsmath}
\usepackage{amssymb}
\usepackage{amsopn}

\newcommand{\qed}{\hfill $\square$}
\DeclareMathOperator{\rng}{rng}
\DeclareMathOperator{\seg}{seg}

\begin{document}
\noindent 
Luke Palmer \\
MATH 4730 \\
2006-04-16

\begin{description}
\item [(11a)] Given a non-empty subset $S$ of $\mathbb{Z}$, consider the
disjoint subsets $S^+ = \{ s \in S: s >= 0 \}$ and $S^- = \{ -s \in S: s
< 0 \}$ covering $S$, and observe that each is order-isomorphic to a
subset of $\omega$.  If $S^+$ is nonempty, then the least element of $S$
is $\min S^+$, otherwise the least element is $-\min S^-$. \qed

\item [(11b)] $E(3) = 3$.  $E(-1) = \omega$.  $E(-2) = \omega+1$.
$\rng E = \omega + \omega$.

\item [(14)] Given $a,b \in A$ such that $a < b$, show $F(a) \subset
F(b)$.  Notice that $a \in F(a)$ but $b \not\in F(a)$, so $F(a) \not=
F(b)$.  If $F(a) = \emptyset$, then $F(a) \subseteq F(b)$.  So assume
that $F(a) \not= \emptyset$ and $a' \in F(a)$.  $a' \leq a < b$, so
$a' \in F(b)$. \qed

\item [(16)] The following three cases follow from Theorem 7K about
well-orderings:
 \begin{itemize}
 \item $\alpha^+ = \seg b$ for some $b \in \beta$: Then $\alpha^+ \in
   \beta$ and therefore $\alpha^+ \in \beta^+$.
 \item $\alpha^+ = \beta$: Then $\alpha^+ \in \beta \cup \{\beta\} =
   \beta^+$.
 \item $\beta = \seg a$ for some $a \in \alpha^+$:  Then $\beta \in
   \alpha^+$, so $\beta \in \alpha \subseteq \beta$ or $\beta = \alpha
   \in \beta$, a contradiction. \qed
 \end{itemize}

\item [(18)] If $\bigcup S =
\alpha^+$ then for every element $s$ of $S$, $s \subseteq \alpha^+$.
Since $s$ is an ordinal, either $s \in \alpha^+$ or $s = \alpha^+$.
Also, $\bigcup S = \alpha^+ \in S$, so $\alpha^+$ is the greatest
element.  This proves part \textit{(i)} by contrapositive, and its
negation trivially implies part \textit{(ii)}.  \qed

\item [(20)] Suppose that $S$ were infinite.  We can construct an infinitely
ascending sequence of $R$ by induction.  However, this sequence will
be an infinitely descending sequence of $R^{-1}$, contradicting its
being a well-ordering. \qed
\end{description}

\end{document}

\documentclass[12pt]{article}
\usepackage{amsmath}
\usepackage{amsthm}

\DeclareMathOperator{\ssup}{ssup}
\DeclareMathOperator{\cf}{cf}
\DeclareMathOperator{\rng}{rng}
\newtheorem*{lemma*}{Lemma}

\begin{document}
\noindent Luke Palmer \\
2006-04-30 \\
MATH 4730: Set Theory

\section*{9.12}

$\cf{0} = 0$, and $0 = \ssup \emptyset$;  $\cf{(\kappa+1)} = 1$, and
$\kappa+1 = \ssup{\{\kappa\}}$, which is clearly of minimal cardinality.

Suppose $\lambda$ is a limit ordinal.  Let $S$ be a set of ordinals
smaller than $\lambda$ of cardinality strictly smaller than $\cf
\lambda$.  Then there exists a $\beta < \lambda$ which is strictly
greater than every element of $S$.  But then $\ssup S$ is at most $\beta
< \lambda$.  Therefore a set with strict supremum $\lambda$ must have
cardinality at least $\cf \lambda$.  A set with size exactly $\cf
\lambda$ exists by the definition of cofinality. \qed

\section*{9.17}

Suppose that there exists an $f: \bigcup\limits_{i \in I} A_i
\xrightarrow{\text{onto}} \bigotimes\limits_{i \in I} B_i$.  Let's view
the elements of $\rng f$ as functions $I \mapsto B_i$.  Create by the
axiom of choice a function $h$ where $h(i) \in B_i \setminus f[A_i]$
(which is always nonempty because $\bar{\bar{A_i}} < \bar{\bar{B_i}}$).
There is no element of $\bigcup\limits_{i \in I} A_i$ which maps to $h$,
so $f$ must not have been onto. \qed

\section*{9.19}

Since $\kappa$ is a regular cardinal, $\cup S < \kappa$ because
$\bar{\bar{S}} < \kappa$.  Since $\kappa$ is a cardinal and thus a limit
ordinal, $V_\kappa = \bigcup\limits_{\theta \in \kappa} V_\theta$.
Therefore there exists a $\theta < \kappa$ where $S \in V_\theta$, so $S
\in V_\kappa$.  \qed

\section*{Plus}

\begin{description}
\item[(i)] A series of informal proofs, because most of the results are
fairly obvious.
 \begin{itemize}
 \item $5 \not\cong [1,5) \cong 4$.
 \item $omega$ is indecomposable, because any smaller number is finite,
 and there are $\aleph_0$-many numbers between any finite number and
 $\omega$.
 \item $\omega+\omega \not\cong [\omega,\omega+\omega) \cong \omega$.
 \item $\omega^2$ is indecomposable, for any smaller number is of the
 form $\omega \cdot m + n$ for finite numbers $m$ and $n$.  Attempting
 to decompose, we find $\omega \cdot m + n + \omega^2 = \omega \cdot m
 + \omega^2 = \omega^2$.
 \item $\omega_1$ is indecomposable, because for any smaller number
 $\kappa$, $[0,\kappa) = \kappa$ has cardinality at most $\aleph_0$, so
 $[\kappa,\omega_1)$ must have cardinality $\aleph_1$, and the order
 type of this range is certainly no larger than $\omega_1$.
 \item $\omega_1 + \omega^2$ is nonindecomposable, because it is
 not isomorphic to $[\omega_1,\omega_1+\omega^2) \cong \omega^2$.
 \end{itemize}
\item[(ii)] Consider the ordinal $\omega_1 \cdot \omega$.  This is
indecomposable for a similar reason that $\omega^2$ is above, and has
all the other desired properties.
\end{description}
\end{document}

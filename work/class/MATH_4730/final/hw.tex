\documentclass[12pt]{article}

\usepackage{amsmath}
\usepackage{amssymb}
\usepackage{amsthm}

\DeclareMathOperator{\cf}{cf}
\DeclareMathOperator{\rnk}{rnk}
\DeclareMathOperator{\rng}{rng}
\DeclareMathOperator{\wcard}{card}

\newcommand{\card}[1]{\bar{\bar{#1}}}

\begin{document}
\begin{description}
\item[(1)]
  If $\kappa$, $\lambda$, and $\mu$ are cardinals, then $\kappa^{\lambda
  + \mu} = \kappa^\lambda \cdot \kappa^\mu$.
  \begin{proof}
	Given $\card{K} = \kappa$, $\card{L} = \lambda$, $\card{M} = \mu$,
	and $L \cap M = \emptyset$.  Given a function $f: L \cup M \mapsto
	K$.  Construct $g: L \mapsto K = f \upharpoonright L$ and $h: M
	\mapsto K = f \upharpoonright M$.  So $\kappa^{\lambda + \mu} \le
	\kappa^\lambda \cdot \kappa^\mu$.

	Given a pair of functions $g: L \mapsto K$ and $h: M \mapsto K$.
	Construct $f: L \cup M \mapsto K = g \cup h$.  $g$ and $h$ have
	disjoint domains, so $f$ is a function.  Therefore $\kappa^\lambda
	\cdot \kappa^\mu \le \kappa^{\lambda + \mu}$, and therefore they are
	equal.
  \end{proof}

\item[(2)]
  $2^{\aleph_0} \neq \aleph_{\omega_1 \cdot 49 + \omega}$
  \begin{proof} 
    Let $\beta = \omega_1 \cdot 49 + \omega$.  $\beta$ is a limit ordinal, 
	so $\cf{\aleph_\beta} = \cf{\beta} = \aleph_0$ (by Enderton's
	Theorem 9N).  However, by K\"onig's theorem, $\aleph_0 <
	\cf{2^{\aleph_0}}$.
  \end{proof}

\item[(3)]
  If $A$ is a set well-ordered by $(<)$ and $f:A \mapsto A$ satisfies $x
  < y \Rightarrow f(x) < f(y)$, then for all $x \in A$, $x \le f(x)$.
  \begin{proof}
	If not, then there is a least $x$ such that $f(x) < x$.  Construct
	by recursion $R:\omega \mapsto A$ where $R(0) = x$; $R(a+1) =
	f(R(a))$.  $R(1) < R(0)$, and given $R(n+1) < R(n)$, $f(R(n+1)) <
	f(R(n))$, so $R(n+2) < R(n+1)$.  Therefore $R$ is an infinitely
	descending sequence in $A$, a contradiction.
  \end{proof}

\item[(4)]
  There exists a sequence of sets $A_0 \supseteq A_1 \supseteq \cdots \supseteq
  A_\alpha \supseteq \cdots$ ($\alpha < \omega_1$) with
  $\bigcap\limits_{\alpha}{A_\alpha} = \emptyset$, and all $A_\alpha$
  uncountable.  
  \begin{proof}
    Namely, let $A_\alpha = \omega_1 - \alpha$.  Every
    $\alpha < \omega_1$ is countable, so subtracting it from uncountable
    $\omega_1$ leaves an uncountable set.  Also, any ordinal $\alpha
    \notin A_{\alpha+1}$, so the intersection is empty.
  \end{proof}
  
\item[(5)]
  For $\lambda$ an infinite cardinal, $2^\lambda = \lambda^\lambda$.
  \begin{proof}
    By the Schr\"oder-Bernstein theorem, $2^\lambda \le \lambda^\lambda
	\le (2^\lambda)^\lambda \le 2^{\lambda \cdot \lambda} \le
	2^\lambda$.
  \end{proof}

\item[(7)]
  Given a set $P$ of $\aleph_0$-many people.  There is an infinite
  subset of that set such that all people have met each other or all
  people have not met each other.
  \begin{proof}
	Let $M \cup \tilde{M} = P^{(2)}$, where $P^{(2)}$ is the set of
	2-element subsets of $P$.  $M$ represents the set of pairs of people
	who have met, and $\tilde{M}$ represents the set of pairs of people
	who have not.  Then by Ramsey's theorem, there exists a $P'
	\subseteq P$ where $\card{P'} = \aleph_0$ and either $P'^{(2)}
	\subseteq M$ or $P'^{(2)} \subseteq \tilde{M}$.
  \end{proof}

\item[(8ai)]
  There is a countable subset of $[0,1]$ isomorphic to $\omega^2$ under
  the usual $(<)$ of the real line.
  \begin{proof}
    Define $v: \mathbb{R} \times \mathbb{R} \times \omega \mapsto
	\mathbb{R}$ by $v(a,b,n) = b - 2^{-n}(b-a)$.  Now define $w:
	\omega^2 \mapsto \mathbb{R}$ by $w(\omega \cdot c + d) = v(1 -
	2^{-c}, 1 - 2^{-(c+1)}, d)$.  The set $W = \rng{w}$ is a subset of
	$[0,1]$ and isomorphic to $\omega^2$.  Since the domain of $w$ is a
	set of ordinals, if $w$ is monotonically increasing then $W$ is
	well-ordered.  Notice that $v(a,b,n) \in [a,b)$ for any $n$ and it
	is monotonically increasing in $n$.  Then it is easy to see that $w$
	is monotonically increasing.
  \end{proof}

\item[(8aii)]
  There is no $X \subseteq [0,1]$ with $X$ well-ordered by the usual
  $(<)$ of the real line and $X$ uncountable.
  \begin{proof}
	Assume that there is.  Let $f: \card{X} \mapsto X$ be the function
	that well-orders $X$ (for example, $f(0) = \text{the smallest
	element in X}$, etc.).  Let $q: \mathbb{R} \times \mathbb{R} \mapsto
	\mathbb{Q}$ be a function that selects a rational between its two
	arguments.  Let $g: \card{X} \xrightarrow{1-1}
	\mathbb{Q}$, by induction: $g(0) = q(f(0), f(1))$; $g(\alpha+1) =
	q(f(\alpha+1), f(\alpha+2))$; $g(\lambda) = q(f(\lambda),
	f(\lambda+1))$.  We have just shown that $\card{X} \le
	\card{\mathbb{Q}}$, contradicting that $X$ is uncountable.
  \end{proof}

\item[(9)]
  $\aleph_\omega^{\aleph_0} > \aleph_\omega$.
  \begin{proof}
    Let $\card{I} = \aleph_0$, and let $A_i$ for $i \in I$ be a bunch of
	disjoint sets each with cardinality $\aleph_\omega$.
	By a theorem due to K\"onig, $\aleph_\omega^{\aleph_0} = 
	\wcard{\bigotimes\limits_{i}{A_i}} > 
	\wcard{\bigcup\limits_{i}{A_i}} = \aleph_\omega \cdot
	\aleph_\omega = \aleph_\omega$.
  \end{proof}

\end{description}
\end{document}

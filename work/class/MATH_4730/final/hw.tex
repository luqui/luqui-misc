\documentclass[12pt]{article}

\usepackage{amsmath}
\usepackage{amssymb}
\usepackage{amsthm}

\DeclareMathOperator{\cf}{cf}
\DeclareMathOperator{\rnk}{rnk}

\newcommand{\card}[1]{\bar{\bar{#1}}}

\begin{document}
\begin{description}
\item[(1)]
  If $\kappa$, $\lambda$, and $\mu$ are cardinals, then $\kappa^{\lambda
  + \mu} = \kappa^\lambda \cdot \kappa^\mu$.
  \begin{proof}
	Given $\card{K} = \kappa$, $\card{L} = \lambda$, $\card{M} = \mu$,
	and $L \cap M = \emptyset$.  Given a function $f: L \cup M \mapsto
	K$.  Construct $g: L \mapsto K = f \upharpoonright L$ and $h: M
	\mapsto K = f \upharpoonright M$.  So $\kappa^{\lambda + \mu} \le
	\kappa^\lambda \cdot \kappa^\mu$.

	Given a pair of functions $g: L \mapsto K$ and $h: M \mapsto K$.
	Construct $f: L \cup M \mapsto K = g \cup h$.  $g$ and $h$ have
	disjoint domains, so $f$ is a function.  Therefore $\kappa^\lambda
	\cdot \kappa^\mu \le \kappa^{\lambda + \mu}$, and therefore they are
	equal.
  \end{proof}

\item[(2)]
  $2^{\aleph_0} \neq \aleph_{\omega_1 \cdot 49 + \omega}$
  \begin{proof} 
    Let $\beta = \omega_1 \cdot 49 + \omega$.  $\beta$ is a limit ordinal, 
	so $\cf{\aleph_\beta} = \cf{\beta} = \aleph_0$ (by Enderton's
	Theorem 9N).  However, by K\"onig's theorem, $\aleph_0 <
	\cf{2^{\aleph_0}}$.
  \end{proof}

\item[(3)]
  If $A$ is a set well-ordered by $(<)$ and $f:A \mapsto A$ satisfies $x
  < y \Rightarrow f(x) < f(y)$, then for all $x \in A$, $x \le f(x)$.
  \begin{proof}
	If not, then there is a least $x$ such that $f(x) < x$.  Construct
	by recursion $R:\omega \mapsto A$ where $R(0) = x$; $R(a+1) =
	f(R(a))$.  $R(1) < R(0)$, and given $R(n+1) < R(n)$, $f(R(n+1)) <
	f(R(n))$, so $R(n+2) < R(n+1)$.  Therefore $R$ is an infinitely
	descending sequence in $A$, a contradiction.
  \end{proof}

\item[(7)]
  Given a set $P$ of $\aleph_0$-many people.  There is an infinite
  subset of that set such that all people have met each other or all
  people have not met each other.
  \begin{proof}
	Let $M \cup \tilde{M} = P^{(2)}$, where $P^{(2)}$ is the set of
	2-element subsets of $P$.  $M$ represents the set of pairs of people
	who have met, and $\tilde{M}$ represents the set of pairs of people
	who have not.  Then by Ramsey's theorem, there exists a $P'
	\subseteq P$ where $\card{P'} = \aleph_0$ and either $P'^{(2)}
	\subseteq M$ or $P'^{(2)} \subseteq \tilde{M}$.
  \end{proof}

\end{description}
\end{document}

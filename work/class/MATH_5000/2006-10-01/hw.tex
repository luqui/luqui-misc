\documentclass[12pt]{article}

\usepackage{amsmath}
\usepackage{amssymb}
\usepackage{amsthm}
\usepackage{mathrsfs}
\usepackage{empheq}

\DeclareMathOperator{\rng}{rng}
\DeclareMathOperator{\dom}{dom}
\DeclareMathOperator{\wcard}{card}

\newcommand{\card}[1]{\bar{\bar{#1}}}
\newcommand{\power}[1]{\mathscr{P}(#1)}
\newcommand{\functionsof}[2]{\;^{#1}\!#2}

\newtheorem*{lemma*}{Lemma}
\newtheorem*{theorem*}{Theorem}

\begin{document}
\noindent Luke Palmer \\
MATH 5000 \\
2006-10-01

\begin{description}
\item[(1)] To define the transitive closure $TC(x)$, define recursively
$g(0) = x$, $g(n+1) = \cup g(n)$ and then take $TC(x)$ = $\bigcup\limits_{n
< \omega}{g(n)}$.

\begin{theorem*}
$TC(x)$ is transitive.
\end{theorem*}
\begin{proof}
Given $b \in a \in TC(x)$.  Then there must be a least $n$ where $a \in
g(n)$.  $b \in \cup g(n) = g(n+1)$, so $b \in \bigcup\limits_{n <
\omega}{g(n)} = TC(x)$.
\end{proof}

\begin{theorem*}
If $Y$ is a transitive set and $X \subseteq Y$, then $TC(X) \subseteq
Y$.
\end{theorem*}
\begin{proof}
$g(0) = X \subseteq Y$.  Suppose $g(n) \subseteq Y$.  Let $x \in
g(n+1)$.  Since $g(n+1) = \cup g(n)$, there is a $x' \in g(n)$ such that
$x \in x'$.  However, $x' \in Y$, and since $Y$ is transitive, $x \in
Y$. 

So all $g(n) \subseteq Y$, so their union $\subseteq Y$.
\end{proof}

\item[(2)] For every ordinal $\alpha$, $\alpha \in V_{\alpha+1} - V_\alpha$.
\begin{proof}
Suppose that for every ordinal $\gamma < \alpha$, $\gamma \in
V_{\gamma+1} - V_\gamma$.

\begin{description}
\item[Case 0:] $\alpha = 0$.  Obvious.
\item[Case 1:] $\alpha = \beta+1 = \beta \cup \{\beta\}$.  $\beta \in
  V_{\beta+1} - V_\beta$.  $\alpha \in V_{\alpha+1}$ because $\beta, \{\beta\}
  \in V_{\alpha} \cup \power{V_{\alpha}} = V_{\alpha+1}$.  
  
  $\alpha \not\in V_{\alpha}$ because $\beta \not\in V_\beta$, so
  $\{\beta\} \not\in V_{\beta+1} = V_{\alpha}$.
\item[Case 2:] $\alpha$ is a limit ordinal.  $V_\alpha =
\bigcup\limits_{\gamma < \alpha}{V_\gamma}$, and and each $\gamma$ is in
$V_{\gamma+1}$, so $\alpha \subseteq V_\alpha$.  Therefore, $\alpha \in
V_{\alpha+1}$.  

Suppose $\alpha \in V_\alpha$.  Then there must be some $\gamma <
\alpha$ such that $\alpha \in V_\gamma$.  However, $\gamma \not\in
V_\gamma$, contradicting $\gamma \in \alpha$.  So $\alpha \not\in
V_\alpha$.
\end{description}
\end{proof}

\item[(3)]
\begin{theorem*}
  $AC \Leftrightarrow$ Given $P$ a partition of $A$, there is a subset
  $X$ of $A$ such that each $x \cap X$ for $x \in P$ is a singleton.
\end{theorem*}

\begin{proof}
\begin{description}
\item[$\Rightarrow$] Pick a choice function $f$ on $P$.  Then define $X
= \rng{f}$.  It is easy to see that this satifies the conditions.
\item[$\Leftarrow$] Given a collection of nonempty sets $\mathscr{A}$.
\textit{Wlg} assume they are disjoint.  Then $\mathscr{A}$ is a
partition of $\cup \mathscr{A}$, so there is an $X \subseteq \cup
\mathscr{A}$ such that each $x \cap X$ for $x \in \mathscr{A}$ is a
singleton.  Then $f(x) = \cup(x \cap X)$ is a choice function on
$\mathscr{A}$.
\end{description}
\end{proof}

\begin{theorem*}
  For any two sets $A$ and $B$, the following are equivalent to the
  axiom of choice:  
    
	(i) Either there is a function $A \xmapsto{1-1} B$ or there
	is a function $B \xmapsto{1-1} A$.
	
	(ii) Either there is a function $A \xmapsto{onto} B$ or there
	is a function $B \xmapsto{onto} A$.
\end{theorem*}
\begin{proof}
 \begin{description}
 \item[(i) $\Rightarrow$ (ii):]  
 Assume \textit{wlg} that there is an $f: A \xmapsto{1-1} B$.  Since $f$
 is 1-1, there is a function $f^{-1}$, which is obviously onto $A$,
 because $A$ was $f$'s domain.
 
 \item[(ii) $\!\wedge\, AC \Rightarrow$ (i):]
 Assume \textit{wlg} that there is an $f: A \xmapsto{onto} B$.  Then
 $f^{-1}$ is a relation with domain $B$.  Pick a choice function $c$ on
 $\power{A}-\{0\}$.  Define $g: B \mapsto A$ by $g(x) =
 c(f^{-1}[\{x\}])$.  $g(b) = g(b') \Rightarrow f(g(b)) = f(g(b'))
 \Rightarrow b = b'$, so $g$ is 1-1.
 
 \item[$AC \Rightarrow$ (ii):] Fix a well-ordering $<_A$ for $A$ and
 $<_B$ for $B$.  Let $a: A \mapsto \alpha$ be the isomorphism from
 $(A,<_A)$ to its unique ordinal $\alpha$.  Likewise $b: B \mapsto
 \beta$.  Assume \textit{wlg} $\alpha \le \beta$.  Define $f: B \mapsto
 A$ by $f(x) = a^{-1}(b(x))$ whenever $b(x) < \alpha$, and $f(x) = a'$
 otherwise, where $a'$ some constant member of $A$.  $f$ is onto $A$.
 
 \item[(ii) $\Rightarrow AC$:] Given any set $A$, let $\alpha$ be its
 Hartog ordinal.  So there is no map from $A$ onto $\alpha$, so there
 must (by \textit{(ii)}) be a function $f: \alpha \xmapsto{onto} A$.
 Now define $g: A \mapsto \alpha$ by $g(x) = $ the least ordinal $\beta$
 such that $f(\beta) = x$.  Then $(A,<_A)$ where $a <_A b$ iff $f(a) <
 f(b)$ is a well-ordering of $A$.
 \end{description}
\end{proof}

\begin{theorem*}
  $AC \Leftrightarrow$ Every linearly ordered subset $L$ of a poset $P$
  can be extended to a maximal $L' \supseteq L$ subset of $P$ which is
  linearly ordered.
\end{theorem*}

\begin{proof}
\begin{description}
\item[$\Rightarrow$] Let $\mathscr{L}$ be the set of all linear
orderings of $P$, and order the orderings by $\subset$.  Every
linearly-ordered (by $\subset$) $L \subseteq \mathscr{L}$ has an upper
bound, namely $\cup L$ (which also must be linearly ordered, thus in
$\mathscr{L}$).  By Zorn's lemma, there is a maximal linear ordering
$L'$ in $\mathscr{L}$, i.e. $L' \supseteq L$ for all $L \in \mathscr{L}$.
\item[$\Leftarrow$] (By showing $\Rightarrow$ (i) above)  Given $A$ and
$B$, let $P$ be the set of all 1-1 function subsets of $A \times B$,
with $f < g$ iff $f \subset g$.  Then, since the empty set is a linear
ordering, there is a maximal linearly-ordered subset $L' \subset P$.
$f = \cup L'$ is a 1-1 function either from $A$ to $B$ or vice versa.  If
its domain weren't (\textit{wlg}) $A$, then we could pick $a \in A -
\dom f$ and $b \in B - \rng f$ and add $\langle a,b \rangle$ to the
ordering, contradicting $L'$'s maximality.\footnote{This one is very
fuzzy, and I'm not sure if I have succeeded in proving this.  I think
the idea of what to do with the ``maximal linear ordering'' has me
confused.}
\end{description}
\end{proof}


\begin{theorem*}
  $AC \Leftrightarrow$ for any $A$, there is a function $f$ with domain
  $\power{A}-\{0\}$ such that $f(x) \in x$.
\end{theorem*}
\begin{proof}
\begin{description}
\item[$\Rightarrow$] Let $f$ be a choice function on
$\power{A}-\{0\}$
\item[$\Leftarrow$]  Given a collection of nonempty sets $\mathscr{A}$.
Let $A = \cup\mathscr{A}$, so that $\mathscr{A} \subseteq \power{A}$.
Then there is an $f$ from above, so we can construct a choice function
on $\mathscr{A}$ by $f \upharpoonright \mathscr{A}$.
\end{description}
\end{proof}
\end{description}

\end{document}

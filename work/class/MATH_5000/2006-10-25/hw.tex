\documentclass[12pt]{article}

\usepackage{amsmath}
\usepackage{amssymb}
\usepackage{amsthm}
\usepackage{mathrsfs}
\usepackage{empheq}

\DeclareMathOperator{\rng}{rng}
\DeclareMathOperator{\dom}{dom}
\DeclareMathOperator{\wcard}{card}
\DeclareMathOperator{\fst}{fst}
\DeclareMathOperator{\snd}{snd}
\DeclareMathOperator{\density}{den}

\newcommand{\card}[1]{\bar{\bar{#1}}}
\newcommand{\power}[1]{\mathscr{P}(#1)}
\newcommand{\functionsof}[2]{\;^{#1}\!#2}

\newtheorem*{lemma*}{Lemma}
\newtheorem*{theorem*}{Theorem}

\begin{document}
\noindent Luke Palmer \\
MATH 5000 \\
2006-10-25

\begin{description}
\item[(Aa)] There is no set of uncountably many nonintersecting circles
in the plane with distinct centers.
  \begin{proof}
  Given a set of nonintersecting circles with distinct centers.  There
  is a set of nonintersecting open sets around the centers of the
  circles.  Pick a rational out of each such set, and you have a 1-1 map
  from circles to $\mathbb{Q}$.
  \end{proof}

\item[(Ab)] There is no set of uncountably many figure-8s in the plane.
  \begin{proof}
  Similar argument as \textbf{(Aa)}, using the crossing at the middle of
  the figure-8 instead of the center.
  \end{proof}

\item[(B)] ...

\item[(C)] The Sheffer Stroke ($p|q$ iff $\neg p \wedge \neg q$) is a
complete logical connective.
  \begin{proof}
  $\neg p$ iff $p|p$.  $p \wedge q$ iff $(\neg p)|(\neg q)$.
  \end{proof}

\item[(D)] Logical equivalence together with negation is not a complete
set of logical connectives.
  \begin{proof}
  I will show that the expression $p \Rightarrow q$ is not representable
  by proving that every truth table in $p$ and $q$ with these
  connectives has an even number of ``true'' values.  By induction on
  $\phi$:

  Trivial for $\phi = p$, $\phi = q$.

  For $\phi = \neg \psi$, then the number of true values in $\phi$ is
  four minus the number of true values in $\psi$, still even.

  For $\phi = \psi \Leftrightarrow \chi$, it suffices to show that the
  number of true values in $\phi$ is even when the number of true values
  in $\psi$ and $\chi$ is exactly two (it is trivial for zero and four).
  Wlg assume that $\psi$'s truth table reads $TTFF$ (we can assume
  negated $p$ and $q$ as appropriate).  Then for each truth table of
  $\chi$:
  \begin{itemize}
  \item $\chi = TTFF$ then $\phi = TTTT$.
  \item $\chi = TFTF$ then $\phi = TFFT$.
  \item $\chi = TFFT$ then $\phi = TFTF$.
  \item $\chi = FTTF$ then $\phi = FTFT$.
  \item $\chi = FTFT$ then $\phi = FTTF$.
  \item $\chi = FFTT$ then $\phi = FFFF$.
  \end{itemize}

  All of which have even numbers of ``true'' values.
  \end{proof}

\item[Ea] 
  \begin{align*}
           & \forall x \forall y \forall z\; (x+y)+z=x+(y+z) \\
	\wedge & \forall x \forall y \forall z\; (xy)z = x(yz) \\
	\wedge & \forall x \forall y\; x+y=y+x \\
	\wedge & \forall x \forall y\; xy=yx \\
	\wedge & \forall x \forall y \forall z\; a(b+c) = (ab)+(ac) \\
	\wedge & \exists x \exists y\; (x \ne y \\
	       & \phantom{\exists x \exists y\;} \wedge \forall z\; x+z=z \\
		   & \phantom{\exists x \exists y\;} \wedge \forall z\; yz=z \\
		   & \phantom{\exists x \exists y\;} \wedge \forall z \exists w\; z+w=x \\
		   & \phantom{\exists x \exists y\;} \wedge \forall z\; (z \ne x \Rightarrow \exists w\; zw=y))
  \end{align*}

\item[Eb]  I'll use symbols other than $x,y,z,w$ as variables for readability.

  Write $(x|y)$ as shorthand for $\exists z\; xz=y$.  Write $P(x)$ as shorthand
  for $\forall y\; ((y|x) \Rightarrow (y=1 \vee y=x))$ (which must only be used
  when the variable $1$ is in scope).  Write $x \le y$ as shorthand for
  $\exists z\; x+z=y$.

  \begin{equation*}
  \begin{split}
    \exists 1 (&\forall x\; 1x=x \\
               \wedge &\forall x \exists y\; (x \le y \wedge P(y) \wedge P((y+1)+1)))
  \end{split}
  \end{equation*}
\end{description}
\end{document}

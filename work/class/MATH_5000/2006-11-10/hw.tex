\documentclass[12pt]{article}

\usepackage{amsmath}
\usepackage{amssymb}
\usepackage{amsthm}
\usepackage{mathrsfs}
\usepackage{empheq}
\usepackage[mathcal]{euscript}

\DeclareMathOperator{\rng}{rng}
\DeclareMathOperator{\dom}{dom}
\DeclareMathOperator{\wcard}{card}

\newcommand{\card}[1]{\bar{\bar{#1}}}
\newcommand{\power}[1]{\mathscr{P}(#1)}
\newcommand{\functionsof}[2]{\;^{#1}\!#2}
\newcommand{\proves}[0]{\vdash}
\newcommand{\elemsubmodel}{\prec}
\newcommand{\submodel}{\le}

\newtheorem*{lemma*}{Lemma}
\newtheorem*{theorem*}{Theorem}

\begin{document}
\noindent Luke Palmer \\
MATH 5000 \\
2006-10-25

\begin{description}
\item[(1)] If $\mathcal{A},\mathcal{B} \elemsubmodel \mathcal{C}$ and
$\mathcal{A} \submodel \mathcal{B}$, then $\mathcal{A} \elemsubmodel
\mathcal{B}$.
  \begin{proof}
  Given a vector $\bar{a}$ of elements of $A$ and a formula
  $\varphi(\bar{a})$.   We have that $\mathcal{A} \models
  \varphi(\bar{a})$ iff $\mathcal{C} \models \varphi(\bar{a})$, since
  $\mathcal{A} \elemsubmodel \mathcal{C}$.  However, since $A \subseteq
  B$ and $\mathcal{B} \elemsubmodel \mathcal{C}$,  $\mathcal{B} \models
  \varphi(\bar{a})$ iff $\mathcal{C} \models \varphi(\bar{a})$.
  Therefore $\mathcal{A} \models \varphi(\bar{a})$ iff $\mathcal{B}
  \models \varphi(\bar{a})$; i.e. $\mathcal{A} \elemsubmodel
  \mathcal{B}$.
  \end{proof}

\item[(3)] A set of setences $\Gamma$ is complete iff any two models of
$\Gamma$ are elementarily equivalent.
  \begin{proof}
  ($\Rightarrow$)  Suppose $\mathcal{A}$ and $\mathcal{B}$ are models of
  $\Gamma$ and $\mathcal{A} \not\equiv \mathcal{B}$.  Then there must be
  some sentence $\varphi$ such that $\mathcal{A} \models \varphi$ and
  $\mathcal{B} \not\models \varphi$.  But if $\Gamma \models \varphi$,
  then $\mathcal{B} \not\models \Gamma$, and similarly if $\Gamma
  \not\models \varphi$ then $\mathcal{A} \not\models \Gamma$, so
  $\Gamma$ must not be complete.

  ($\Leftarrow$)  Suppose $\Gamma$ is not complete.  Then there is some
  $\varphi$ such that neither $\Gamma \models \varphi$ nor $\Gamma
  \models \neg\varphi$.  Thus there must be a model $\mathcal{A} \models
  \Gamma \cup \{\varphi\}$ and a model $\mathcal{B} \models \Gamma \cup
  \{\neg\varphi\}$.  Clearly $\mathcal{A} \not\equiv \mathcal{B}$.
  \end{proof}

\item[(4a)] A set of sentences $\Sigma \models \varphi$ iff $\Sigma \cup
\{\neg\varphi\}$ is inconsistent.
  \begin{proof}
  ($\Leftarrow$) Suppose $\Sigma \cup \{\neg\varphi\}$ has no models.
  Then either $\Sigma$ has no models or it has a model $\mathcal{A}$.
  If it has none, then we're done.  It must be the case that
  $\mathcal{A} \models \varphi$ because of our assumption.

  ($\Rightarrow$) Suppose a structure $\mathcal{A} \models \Sigma$ and
  $\Sigma \models \varphi$.  By definition $\mathcal{A} \models
  \varphi$, so it is impossible that $\mathcal{A} \models \neg\varphi$.
  \end{proof}

\item[(5)] If $\mathcal{U}$ is an ultrafilter, $Y \in \mathcal{U}$, and
$Y = \bigcup\limits_{i < n}{Y_i}$, then $Y_i \in \mathcal{U}$ for some $i$.
  \begin{proof}
  Suppose not.  Then $Y \in \mathcal{U}$ and $\bar{Y_i} \in \mathcal{U}$
  for every $i < n$.  Then the intersection $Y \cap \bigcap\limits_{i <
  n}{\bar{Y_i}}$ would be empty.
  \end{proof}
\end{description}
\end{document}

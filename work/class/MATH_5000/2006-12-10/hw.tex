\documentclass[12pt]{article}

\usepackage{amsmath}
\usepackage{amssymb}
\usepackage{amsthm}
\usepackage{mathrsfs}
\usepackage[mathcal]{euscript}

\DeclareMathOperator{\rng}{rng}
\DeclareMathOperator{\dom}{dom}
\DeclareMathOperator{\wcard}{card}
\DeclareMathOperator{\theory}{Th}

\newcommand{\card}[1]{\bar{\bar{#1}}}
\newcommand{\power}[1]{\mathscr{P}(#1)}
\newcommand{\functionsof}[2]{\;^{#1}\!#2}
\newcommand{\proves}[0]{\vdash}
\newcommand{\elemsubmodel}{\prec}
\newcommand{\submodel}{\le}

\newtheorem*{lemma*}{Lemma}
\newtheorem*{theorem*}{Theorem}

\begin{document}
\noindent Luke Palmer \\
MATH 5000 \\
2006-12-10

\begin{description}
\item[(2)] If $\tau$ is a term containing no variables, then for some
unique $k \in \omega$, $A \proves \tau = \bar{k}$.
  \begin{proof}
  \textit{By induction on $\tau$}.
  \begin{itemize}
    \item Case $\tau = \bar{0}$.  $A \proves \bar{0} = \bar{0}$.
	\item Case $\tau = S\varsigma$.  By IH, there is a $k \in \omega$
	with $A \proves \varsigma = \bar{k}$.  Therefore, $A \proves
	S\varsigma = S\bar{k}$ by logic (no axioms in $A$ necessary), i.e.
	$A \proves \tau = \overline{k+1}$.
	\item Case $\tau = \rho + \varsigma$.  By IH, there are $j,k \in
	\omega$ with $A \proves \rho = \bar{j}$ and $A \proves \varsigma =
	\bar{k}$.  Thus $A \proves \tau = \bar{j} + \bar{k}$, and we have
	already shown (in a homework long ago) that these axioms give the
	correct result.
	\item Case $\tau = \rho \cdot \varsigma$ and $\tau = \rho^\varsigma$
	proceed similarly to above.
  \end{itemize}

  Uniqueness follows trivially, since if $A \proves \tau = \bar{j}$ and
  $A \proves \tau = \bar{k}$ with $j \ne k$, then $A \proves \bar{j} =
  \bar{k}$ which we have already shown impossible.
  \end{proof}

\item[(14)] If $f: \omega^n \mapsto \omega$ is a recursive function then
  there is a formula $\phi(\vec{x}, y)$ such that if $f(\vec{k}) = r$ then
  $A \proves \forall y [\phi(\overline{\vec{k}}, y) \Leftrightarrow y =
  \bar{r}]$.
  \begin{proof}
  Since $f$ is recursive, it is defined by a (recursive) formula
  $\theta(\vec{x}, y)$ such that $A \proves \theta(\vec{x},y)$ iff
  $f(\vec{x}) = y$.  Consider the formula $\varphi(\vec{x},y) =
  \theta(\vec{x}, y) \wedge \forall z<y: \neg \theta(\vec{x},z)$.  Note
  that $\varphi$ is recursive, since it is built up only from the
  recursive $\theta$ and bounded quantification.

  If $f(\vec{k}) = r$ then $A \proves \varphi(\vec{k}, r)$, since
  $\theta$ holds there and $\varphi$ is recursive.

  $A \proves \forall y < r: \neg \varphi(\vec{k}, r)$, since
  $\theta$ does not hold for any $y < r$ and that formula is recursive.
  $A \proves \forall y [r < y \Rightarrow \neg \varphi(\vec{k}, r)]$,
  since we can use the proof of $\varphi(\vec{k}, r)$ as its
  counterexample.  Then using (L3) (trichotomy), $A \proves \forall y [y
  \ne r \Rightarrow \neg \varphi(\vec{k}, r)]$, and thus $A \proves
  \forall y [\varphi(\vec{k}, y) \Leftrightarrow y = r]$.
  \end{proof}
\end{description}
\end{document}

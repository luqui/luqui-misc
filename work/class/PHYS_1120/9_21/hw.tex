\documentclass[12pt]{article}

\usepackage{amsmath}

\begin{document}
\noindent Luke Palmer \\
Physics 1120 \\
9/21/2004 Long Answer

\section*{Part 1}

Given an infinitely long cylinder with uniform charge density $\rho$,
what is the electric field at some radius $r$ from the center line
within the body of the cylinder?

We'll use Gauss's law to solve this.  First, observe the lengthwise
symmetry of the problem.  For every charge, there is another charge to
cancel all nonradial components of the resulting field.  We have a
radial electric field.

Also note that the cylinder has no notion of angular orientation about
its axis---it is symmetrical about its axis---so it is only reasonable
for nature to have the magnitude of the electric field equal at each
$r$.

We'll choose a Gaussian surface such that $\cos{\theta}$ is either one
or zero everywhere on the surface (where $\theta$ is the angle between
the radial electric field and the area vector of the surface at that
point).  An approprate surface is a cylinder with length $L$ and radius
$r$ (since we're interested in the field at $r$).

Gauss's law states:

\[
    \epsilon_0 \oint_S \mathbf{E} \cdot \mathbf{dA} = Q
\]

Where $Q$ is the charge enclosed by the Gaussian surface $S$.  

The charge enclosed by our cylinder is $\pi r^2 L\rho$.  So we have:

\[
    \epsilon_0 \oint_S \mathbf{E} \cdot \mathbf{dA} = \pi r^2 L\rho
\]

Since the dot product is zero along the flat edges of our Gaussian
surface (because the field is radial), we can consider only the curved
part of the cylinder.  The magnitude of the electric field along that is
constant, and $\cos{\theta}$ is always one, so:

\begin{align*}
    \epsilon_0 E \oint_S dA &= \pi r^2 L\rho \\
    2 \epsilon_0 E \pi r L  &= \pi r^2 L\rho \\
    E                       &= \frac{r \rho}{2 \epsilon_0}
\end{align*}

Notice that the result is independent of the length of the Gaussian
surface, as well as the radius of the entire cylinder.  The electric
field depends only on the charges whose radius is less than $r$. 

\section*{Part 2}

Halliday, Resnick, and Walker derive that the electric field at a
distance $r$ away from an infinitely long line of charge with uniform
linear charge density $\lambda$ is:

\[
    E = \frac{\lambda}{2\pi\epsilon_0 r}
\]

In order to see the similarity to the result that we just derived, we
need to redefine a couple of things.  First, we need two radii: the
distance at which we are trying to find the electric field $r$, and the
radius of the cylinder $R$.  Going back to our equation:

\begin{align*}
    \epsilon_0 E \oint_S dA &= \pi R^2 L\rho \\
    2 \epsilon_0 E \pi r L  &= \pi R^2 L\rho \\
                 E          &= \frac{R^2 \rho}{2 \epsilon_0 r}
\end{align*}

We can define that the linear charge density $\lambda = \pi R^2\rho$
(each $\lambda$ refers to a circular slice of the cylinder).  Then we
simplify to:

\[
    E = \frac{\lambda}{2 \pi \epsilon_0 r}
\]

Which agrees.

We must be careful to state that this is only when $r > R$, because
otherwise our definition of $\lambda$ is incorrect.  When $r < R$,
$\lambda = \pi r^2\rho$ instead.

\end{document}

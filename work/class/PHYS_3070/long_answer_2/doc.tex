\documentclass[12pt]{article}

\begin{document}
\noindent Luke Palmer \\
2005-01-25 \\
Long Answer no. 2 \\

\begin{description}
\item (1) Q+P Ch. 1, \#6

Having experienced the painful depletion of petroleum, we hope that the
human race would realize that the remaining nonrenewable resources are
disasters waiting to happen.  We are left with three renewable
resources: solar, geothermal, and tidal.  The most promising of the
three is the former, coming in many forms.  It is very expensive
(comparing to fossil fuels in present day) to harvest these sources of
energy, but resource supply is no longer a controlling factor in the
price; leaving only ever-incresing technology.  This means that
most renewable resources get cheaper as their use is widened.

\item (2) Q+P Ch. 2, \#3

15--20\% of the oil in a well either comes to the surface due to natural
geological pressures or can be easily pumped to the surface.  After
that, it is possible to pump water or gas into the well in order to
force the oil into a recoverable position, yielding approximately
another 15\%.  That leaves 65--70\% to be recovered by more complex and
expensive means.

\item (3a)

US domestic oil production reached its peak in 1970.  It is predicted
that global oil production will peak somewhere between 2010 and 2025.

\item (3b)

In 1970, the US was producing approximately 3.2 billion barrels per
year.  Edwards (1993) forecasts that at the world's peak oil production,
the total yield would be between 27 and 36 billion barrels per year.

\item (3c)

(3a) is not uncertain.  The oil production in a given year is a
measurable quantity, and it clearly showed its peak in 1970.  (3b) is
much more uncertain, and we have to consider carefully the method used
by the party calculating it.

\item (4)

The article predicts that there are 487 billion barrels yet to be found.
If this is larger by 25\%, then there are 120 billion barrels on top of
that.  At the current rate of 60 million barrels, this amounts to:

\[
    \frac{1.2 \times 10^11 \, \mathit{bbl}}%
         {6 \times 10^7 \, \mathit{bbl}/\mathit{day}}
  = 2000 \, \mathit{days} = 5.5 \, \mathit{years}
\]

These ``gobs of oil'' don't last us very long at all.

\item (5)

I would imagine the figure to be high.  I trust whatever scientific
reasoning used to come up with a figure, but I don't trust that the
figure hasn't been altered by the oil companies reporting this figure
(assuming that it came from somewhere therein).  The last thing a
producer wants to convince its customers is that they need to start
thinking about switching to a new product.  In particular, if the oil
companies can convince its primary customers, presumably between the age
30 and 50, that the end is after their lifetime, then the threat becomes
much less personal, and people will continue to pay for oil instead of
researching alternate means, which would result in loss of profit for
the oil company.

That's my first-order approximation.  Undoubtedly, there are some deeper
economic mechanics going on that I don't have enough background to
understand.

\end{description}

\end{document}

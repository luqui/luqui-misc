\documentclass[12pt]{article}

\begin{document}
\noindent Luke Palmer \\
2005-02-02 \\
PHYS 3070 Week 3 \\

\begin{description}

\item[(1)] 9800 tons per day isn't a large raw volume of shale, since in
1990 the US was producing 2.7 million tons of coal per day.  Assuming 30
gallons of oil per ton as stated, it is only 15\% as efficient as coal,
\textit{ceteris paribus}.  It is reassuring to know that there is a
reserve of petroleum, so that the eventual depletion of oil from
traditional sources does deprive us of other petroleum products.  It may
not be economic to mine shale for energy, but there is enough oil to
make all the plastics and lubricants we'll need for a long time.

\item[(2)] According to the Energy Information Administration of the
United States\footnote{\texttt{http://www.eia.doe.gov/emeu/cabs/usa.html},
January 2005}, the US has 21.9 billion barrels of proved oil reserves.
From the same source, the US consumed an average of 20.4 million barrels
per day (7.5 billion barrels per year).  On those numbers alone (not
incorporating estimated unproven oil), the US could sustain itself at
its current rate of consumption for about three years.  It's a good
thing we're not trying to be self-sustained.

\item[(3)] Even the lower bound of 600 billion barrels is a lot compared
to the $Q_\infty$ for the US, 280 billion.  Of course, even the upper
bound of 2000 billion barrels isn't enough to sustain indefinitely, but
if technology were to advance enough, we could postpone the oil
depletion crisis for many years.  However, the current state of
technology is such that it is not economic to obtain oil from oil shale,
taking nearly as much energy put into extraction as is extracted.

But if a technology that could efficiently extract oil from oil shale
were developed, it would probably cost a lot more than current petroleum
extraction costs.  This would give us a long transition period, while
increasing the cost of oil to the point where research into alternative,
renewable energy would become feasible in the short term.  Such a
scenario gives us a much gentler landing for the oil crisis, and on
those grounds, it is worthwhile to research oil shale extraction
technology.

\end{description}
\end{document}

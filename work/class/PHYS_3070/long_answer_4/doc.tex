\documentclass[12pt]{article}

\begin{document}

\noindent Luke Palmer \\
2005-02-09 \\
PHYS 3070

\begin{description}
\item[(1)] I looked around mostly at the Alternative Energy Sources
section, because that's what our final project is most concerned with.
In particular, I found a great deal of useful information (which I added
to our project wiki\footnote{\texttt{http://luqui.org/3070}}) at the
EERE (US Department of Energy Office of Energy Efficiency and Renewable
Energy).  However, either the information was too superficial: every-day
descrptions of what's happening without numbers, or too technical:
measurements about the underlying hardware.  I wasn't able to find any
data about efficiency in terms that I understand.

I also looked at How Stirling Engines Work, since I am interested in the
mechanics of power generation.  I understand a little bit better now.

\item[(2)] \textbf{Summary of the Big ideas in 3070:}
\begin{itemize}
\item Our conventional energy supply is very limited.  There will have
to be a change within our lifetimes, according to every reasonable
estimate.

\item There are many alternatives, but none are nearly as cheap as
petroleum.  An energy-demanding society will probably have to use many
of these alternatives simultaneously.

\item Understanding ``big numbers'' and measures of energy is one of the
most important things about understanding the current issues.  Many
journalists and others throw these numbers around (e.g. ``\textit{fifty
million barrels!}'') without understanding what they mean.

\item Order-of-magnitude estimates can get you close enough to the real
numbers that you can double-check any data given to you for its
validity.
\end{itemize}
\end{description}

\end{document}

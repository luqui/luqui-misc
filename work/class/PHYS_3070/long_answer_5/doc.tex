\documentclass[12pt]{article}

\begin{document}
\noindent Luke Palmer\\
02-16-2005\\
PHYS 3070

\begin{description}
\item[(1)] My grandparents kept their florescent lights on in their
house all night.  My grandfather told me that they used less energy to
turn on than to keep them on for a day.  Prof. Pollock and I did a quick
mental check while chatting one day (how many amps can a circuit stand,
how much energy are they releasing in light), and it seemed highly
unlikely that that was the case.  

I've heard that nuclear power is an effectively infinite resource.
People just throw $E = mc^2$ around with the most basic understanding.
For instance, for a $1 \textit{kg}$ glass of wather, you could get $E =
1\,\mathit{kg} \cdot 9 \times 10^16\,m/s = 10^17 J = .08 QBTU$ of
energy.  From one glass of water, that's an awful lot.  But to get
\textit{all} the atomic energy out, completely another story.

I would classify both of these as urban myths, the latter being slightly
more rigorous (but still na\"ive).

\item[(2)] The pump indeed can deliver more \textit{total} energy than
is drawn from the power line.  But the energy equivalent of the
\textit{difference} between the two reservoirs must be less than or
equal to the amount of energy taken from the line.  The extra energy
comes from the heat energy in the cold bath.

\item[(3)] Gasoline engines drive pistons by manually igniting fuel when
the piston is high, causing it to expand and drive the piston back down.
Diesel engines operate at higher peak pressures than gasoline engines,
and utilize that pressure to ignite the fuel at the proper time.  Diesel
engines are up to 50\% more efficient than gasoline engines (20\% to
30\%), but they cannot handle a quick demand for power.  Diesel is also
more energy-rich fuel than gasoline (albeit heavier).

Gas turbines are a ``continuous'' form of a gasoline engine.  They burn
fuel to turn a turbine, and shoot the waste heat out the back.  A gas
turbine is not as efficient as a steam turbine, but its waste heat comes
in a much more useful form, making cogeneration easier.  Its development
has taken some time, because it operates at very high temperatures and
pressures.

\end{description}

\end{document}

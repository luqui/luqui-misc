\documentclass[12pt]{article}

\begin{document}
\noindent
Luke Palmer \\
2005-03-02 \\
PHYS 3070

\begin{description}
\item[(1)] I don't understand this question.  I was under the impression
that I did feel heat from a fireplace when I placed my hand behind a
sheet of glass.  The only remark that I can think of is that the
fireplace is not sending out very much of its energy in the form of
electromagnetic radiation.  Air circulates toward the fireplace, heats
up, and then circulates away from the fireplace, without ever involving
a photon. 

\item[(2)] I looked at the Solar-Electric company's 40 watt solar
panel\footnote{\texttt{http://store.solar-electric.com/kc-40.html}},
which is 20.7" by 25.7" (0.35 square meters), and costs \$220.
Remembering that solar power goes for about \$5 per watt, this sounded
about right.  From this data, I computed that these solar panels would
deliver 115 watts per square meter, and therefore, a kilowatt would
require about 17 square meters.  I figured that their 40 watt quote
would be about as high as it ever produces, which would probably occur
at noon on a clear day in Boulder (in the Summer).

\item[(3a)]
\texttt{fypower.org}\footnote{\texttt{http://www.fypower.org/res/energy.html}}
quotes that the typical California household uses 5,914 kWh annually in
electricity.  That's about 16 kWh per day, or just 650 watts.  We're
looking at solar power, so let's just consider the eight hour day.
During that time we have a demand of about two kilowatts (3 times 650).

\item[(3bc)] I picture a mirrored paraboloidal dish that has water
running through its focus, boiling into steam and being piped back out.
The book states that during the 8 hour day, the sun delivers 600 watts
per square meter, or 0.6 BTU per second per square meter in average
conditions and location.  If we had, say, 20 square meters, then would
have 12 BTU per second, and we could heat a gallon of water by 12
degrees F every second.  Let's say we heat a quarter gallon of water by
96 degrees F every two seconds instead, on a hot summer day (80 degrees
F), giving us a hot bath of 353 K and a cold bath of 300 K, and a carnot
efficiency of 15\%.

\item[(3d)] A 20 square meter dish would, based on previous numbers,
give us 1800 watts, ideally.  I was going to use the system's
approximate linearity and use the 20 square meters to come up with the
actual size, but 20 square meters looks about right.  We've computed the
heat engine ideally, however, as well as used some other optimistic
values, so let's very roughly compensate by a factor of 3/2: 30 square
meters.  This is not inconceivable, but it is not small by any means.
It would probably fit in my back yard.  

We computed in question (2) that 17 square meters would provide our 2
kilowatts using a photovoltaic system.  That would cost about \$10,000
to deploy, however.  Such a heat engine might (and certainly might not)
cost less, even though it takes up more room.  I don't have the
slightest idea how to compute an estimate for the cost of such a thing.
It has the same disadvantages as photovoltaics: it doesn't work in the
nighttime, and its peak load is not very high. 

\end{description}
\end{document}

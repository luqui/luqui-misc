\documentclass[12pt]{article}

\begin{document}
\noindent
Luke Palmer \\
2005-03-09 \\
PHYS 3070

\begin{description}
\item[(1)] Lake Mead Recreational Area's website\footnote{\texttt{
http://www.desertusa.com/colorado/lm\_nra/hoover/du\_hoover.html}} says
that the Hoover Dam generates four billion kWh of energy annually, so it
is about a 500 MW plant.  The same site says that that supplies ``enough
for 500,000 homes.''  My electric bill is about \$100/month, so I use
about 1,000 kWh per month, or 12,000 kWh per year.  500,000 of my houses
would use six billion kWh per year, so the figure is about right.  If
the Hoover dam sells electricity for 10 cents per kWh, they would
collect four hundred million dollars per year.

\item[(2a)] \[
    g(1 \mathit{kg})(30m)(0.9) = 300 J
\]
\item[(2b)] \[
    (2 Cal)(0.03) = 250 J
\]
\item[(2c)] One kilogram of water produces approximately the same amount
of energy in both cases.  Of course, one type of plant is not better
than the other, because hydroelectric plants make use of height
differences of water, while OTEC plants would make use of temperature
differences without height difference.

\item[(3b)] A 1 cent per gallon per month increase in gas tax.  It is
currently not economic to build large scale alternative energy projects.
But such a tax (and a corresponding tax for other petroleum-based
energy) would make it economic in five to ten years, which is about the
time scale for these energy projects.  The developers can predict this
(as long as the tax states openly that it will be increasing over time)
and thus deem it worthwhile to start investing in such a project now.

I have a friend who is working with some politicians to try to implement
this in Colorado.

On a personal scale, the government could subsidy the installation of
solar panels by, say, paying some percentage of the installation cost.
They could introduce new subsidies on wind power, taking it down from
ten cents per kWh to 9.5.  Ultimately, the more customer base there is
for a renewable energy source, the more inspired private research will
be to work on it.

\end{description}
\end{document}

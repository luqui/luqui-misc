\documentclass[12pt]{article}

\begin{document}
\noindent Luke Palmer \\
2006-09-05 \\
PHYS 4810

\subsection*{How we teach and how students learn --- A mismatch
             (McDermott L. AJP 61(4), 1993, p295)}


This paper discusses learning difficulties common in introductory
physics students.  It provides several concrete examples which are easy
to understand, and it is usually clear how to address those issues.  The
paper cites several findings which are vague and difficult to interpret
(the questions below mention a couple of these).  The paper's general
points are forehead-slappingly good, and contrast the way physics is
usually taught.  It stressed the importance of teaching reasoning skills
in addition to physics knowledge, but unfortunately it didn't strongly
address what methods could be used to do this.

\addvspace{1ex}
\noindent \textit{Questions:}

\begin{itemize}
\item The impulse-momentum/work-energy example had me confused for a
little while.  They applied the same force over the same distance to two
different objects, and asked to compare their kinetic energies and
momenta.  It was obvious that their kinetic energies were equal, however
I assumed that this implied their momenta were also equal.  Only after
some calculations did I realize that this wasn't the case.  And I'm
supposed to know this stuff.  That wasn't much of a question, but it is
food for thought/discussion.
\item ``Some seemed to treat the symbol `=' as if it represented only a
mathematical relationship in which the variables may take on any values,
provided the equality is maintained.''  What does this mean?  How is
this erroneous?   As a side note, I heard from one of my friends the
other day that a common technique taught for interpreting word problems
is to replace the word ``is'' with ``='', which I had never heard and
sounded absurd to me. 
\item ``... the instructor must insist that students \textit{confront}
and \textit{resolve} the issue.''  How exactly is this done?  The
``issue'' may not even be apparent to the student.
\end{itemize}


\subsection*{Learning to think like a physicist: A review of
research-based instructional strategies (Van Heuvelen, A. Am. J. Phys.
59, 1991, p891--897)}

This paper identifies the problem in physics education that students are
too passive in their learning experience.  It makes the excellent
analogy that when a student learns a musical instrument or a sport,
individual skills and techniques are taught and practiced independently,
whereas in physics the whole method is demonstrated and then asked to be
repeated a couple days later.  The paper is generally insightful, but
lacks in examples and tends to speculate.  As an added bonus, there is
a hillarious terrible computer analogy, which is repeated below.

\textit{Questions:}

\begin{itemize}
\item ``[Students] believe in impetus forces, \textbf{ma} forces, the
force of inertia, and the force of momentum.''  To students, what is an
impetus force or an \textbf{ma} force?  It is hard for me to reason
about misconceptions given only their misnames.
\item ``A computer would become very confused if given two conflicting
sets of operating instructions, one stored many times over a period of
years and the other provided a few times 1- or 2-week period.  If we
want to the computer to assimilate a new operating system, we must bring
up the old system and dump it before the new system can be saved.
Having both operating systems present at the same time leads only to
confusion.''  Hahahahahaha!  I'm not counting that as a question; it was
just too funny to leave unnoticed.
\item I wouldn't consider a ``hierarchical chart'' the way to build a
framework.  Even if this is how a mental framework is ideally organized
(I don't think it is for me, at least), presenting the chart probably won't
accomplish anything.  ``Yes, the stuff you're learning really is all
unified, I promise.''  Supposing that a hierarchy is the way we'd like
the students to organize their knowledge, what would be a good way to
construct it?  Supposing we can ``tell'' students something to help
construct a framework, what might we tell them (as opposed to having
them do an activity)?
\item The ``system physics'' model seems like something that is
generally true, but may be overgeneralizing.  It depends on the details
of how it is taught; for example we wouldn't want students adding energy
to force to see if they sum to zero.  Overgeneralizing is not
necessarily bad in a pedagogical context, but it strikes me more as a
``rule of thumb'' than a ``mental framework'' (much like ``replace is
with ='').  What are the details of this pedagogy?  What are possible
negative effects of this approach?
\end{itemize}

\end{document}

\documentclass[11pt]{article}

\usepackage{fancyhdr}
\usepackage{gensymb}
\usepackage{amsmath}

\lhead[\fancyplain]{Luke Palmer}
\chead[\fancyplain]{2006-09-13}
\rhead[\fancyplain]{PHYS 4810}
\pagestyle{fancyplain}

\begin{document}
Chapter 24 in HRW 5th edition covers Gauss's law.  They start by
introducing Gauss's law as a technique which can take advantage of
symmetry of problems.  Flux is defined first using the Riemann sum $\Phi
= \sum \mathbf{E}\cdot\mathbf{\Delta A}$.  I was thinking, is this
supposed to help people who don't understand integrals?  What the heck
does $\mathbf{\Delta A}$ mean?  A change vector, even though it is
hardly representing a change so much as a finite infinitesimal (!).  I
don't see the utility of this representation. Anyway, they go on to
define flux in the integral form and then state Gauss's law in terms of
that, followed by a few derivations of things we already knew and some
new problems.

\section*{3}
An electric flux of $\Phi = \# Nm^2/C$ passes through a flat horizontal
surface that has an area of $A = \# m^2$.  The flux is due to a
uniform electric field.  What is the magnitude of the electric field if
the field points $\theta = \# \degree$ above the horizontal?

I used limit cases to figure out which trig function to use.  When the
E-field is horizontal, there is no flux so there is no way to tell what
the E-field is; i.e. we are dividing by 0, so we are dividing by $\sin$.
Then we have to incorporate the area, so I used dimensional analysis to
figure out whether to multiply or divide: divide.  The final formula
comes out to
\[ \frac{\Phi}{A \sin{\theta}} \]

I'll coin the method I used ``speed solving''.  Essentially, I quickly
scanned the problem to get an image so I can determine the \textit{type}
of operations I'll be doing (in this case, multiplying/dividing and some
trig) and then use the easiest-to-think-about cues to trim the branches
of what the solution could be until there is only one.

Mathematically, this problem draws on the definition $\Phi =
\mathbf{E}\cdot\mathbf{A}$ and the well-known definition
$\mathbf{E}\cdot\mathbf{A} = EA\cos{\theta}$.  However, I used neither
of these things formally; instead I used a visual conception of flux and
a good intuitive understanding of the role of trig functions.
Interestingly, the only thing we needed to know about an E-field was
that it is a vector field.

I'd say this is a mediocre problem.  It's good from the standpoint that
it's a little bit unusual in that you have to divide by a trig function.
I liked the hint: ``It's easy to get confused about which angle to use.
Draw a diagram.''  However, it's kind of a plug-and-chug problem, as is
demonstrated by the ``speed solving'' method I used.  It wasn't given
real-world context and you do the standard ``multiply the variables
together with a trig function'' solution.  A student who doesn't
understand isn't likely to make a discovery with this problem (do
most students usually have ephiphanies while solving problems?).

\section*{4 \& 5}

A cubic cardboard box of side $a = \# m$ is placed so its edges are
parallel to the coordinate axes\footnote{That wording is a little
strange; I would place the coordinate axes so that they are parallel to
the edges of the box \texttt{:-)}}.  There is \textbf{NO} net electric
charge inside the box, but the space in and around the box is filled
with a nonuniform electric field of the following form:
$\mathbf{E}(x,y,z) = Kz\mathbf{j} + Ky\mathbf{k}$, where $K = \# N/Cm$.
What is the electric flux throught the top face of the box (where $z =
a$).

I looked at the problem and tried to visualize the electric field with
my fingers.  That really didn't help my visualization, so I continued
with math.  I set up the integral over the top face and evaluated:

\begin{align*}
\int_0^a\int_0^a{(Kz\mathbf{j} + Ky\mathbf{k})\cdot\mathbf{k}\,dy\,dx}
&= \int_0^a\int_0^a{Ky\,dy\,dx} \\
&= \int_0^a{\frac{1}{2}Ka^2\,dx} \\
&= \frac{1}{2}Ka^3
\end{align*}

I checked my units then plugged in the numbers.

I'm now noting from the hint that I didn't even think in terms of
$\mathbf{dA}$; I automatically substituted $\mathbf{k}\,dy\,dx$.  I'm
not sure how relevant that is.

For problem 5, Gauss's law told me to negate my answer to problem 4.
That was fun.

Problem 4 drew on the same physics concepts as problem 3, but the
problem is in a slightly more sophisticated setting due to the integral.
In particular, folks with a weak calculus background will probably have
more trouble here than with problem 3, especially if we are talking in
terms of $\mathbf{dA}$ and whatnot.

Especially in combination with problem 5, this is a good problem.  At
least, to me it was a good problem.  It exercised integrals which feels
more like a nontrivial application of what we're learning; i.e. more
than just regurgitation.  However, the integral was pretty easy, so it
didn't get bogged down in a lot of sticky calculus.  Problem 5 was one
of those problems that could be really hard or really easy depending on
how you approach it.  For me, whenever I see a problem that would be
\textit{that} tedious, it forces me to think if there's a clever way to
do it.  The problem is of that nature that (hopefully) encourages
students to think outside the box by thinking about the charge inside
the box.

\section*{12}

(The problem relies heavily on the diagram, so I won't repeat it)

To solve this problem, I recalled what happened in this case because I
had figured it out before.  However, supposing I hadn't, this would be a
heavy Gauss's law problem.  For a (inside the shell), consider the
sphere at a's radius surrounding.  We can assume the problem is
symmetrical, and there must be some flux (because there is charge inside
the sphere), so the E-field must not be zero.

For the inner surface of the shell, consider the sphere inside the
conductor.  The E-field must always be zero, so there is no flux.
Therefore, there must be no net charge inside the sphere, and since this
is true at all radii inside the conductor, the only place it could go
would be on the inner surface.

For c (inside the conductor), the field is zero because it's a
conductor.

For e (outside the shell), consider the sphere at e's radius from the
charge in the center.  The net charge in the sphere in nonzero, so there
must be flux.  Since the problem is symmetrical, we can argue that the
E-field ought to be distributed equally along that sphere, so the field
is nonzero there.

The whole shell problem is, in my opinion, an excellent exercise in
physics reasoning.  It gives the student who solves the problem a taste
of the power of Gauss's law.  Sometimes you have to use Gauss's law to
find the field based on the enclosed charge, sometimes you have to use
it the other way around.  However, there aren't a variety of ways to
solve it, which can be a weakness.  If a student doesn't grok Gauss's
law, there isn't much he can do.  This may also be a strength in that it
forces the student to ask for help.

\end{document}

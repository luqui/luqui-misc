\documentclass[12pt]{article}

\usepackage{fancyhdr}
\usepackage{gensymb}
\usepackage{amsmath}

\lhead[\fancyplain]{Luke Palmer}
\chead[\fancyplain]{2006-09-18}
\rhead[\fancyplain]{PHYS 4810}
\pagestyle{fancyplain}

% abuse of titley stuff
\title{Discussion Notes}
\author{Seymour Papert: Situating Constructionism}
\date{}

\begin{document}
\maketitle

Take the last 10 minutes of my segment to form the students into groups
and have them design a constructionist lab about voltage.  Pretend that
they have full reign over the curriculum, but they still must be
practical.  The groups should quickly share their ideas at the end of
the segment.  On Thursday, they will get to make another pass, revision,
or rework on their idea.

This should end up being a dumbed-down version of a constructionist
activity, by my take.   It has a long-term nature, it is public so that
students can get ideas from each other, it is building something.  The
only thing that it will miss will be learning content \texttt{:-)}.
However, that may come, too, as they read and think about voltage for
Thursday.

\begin{itemize}
\item Constructionism isn't simply a new teaching methodology.  It
entails changing both how and \textit{what} students learn.  ``... to
break the sense of necessary connection between improving learning and
improving teaching.''  Indeed, so how can we incorporate constructionism
into a traditional classroom, since our institutions won't go for
radical change?

\item ``A variant of this strong claim is that [constructionism] is the
only framework that has been proposed that allows the full range of
intellectual styles and preferences to each find a point of
equilibrium.''  Is that really the case?  I am an example of a person
who best absorbs mathematics by being lectured from a person (as opposed
to a book).  What does a constructionist math environment have for me?

\item ``[Building knowledge structures] happens especially felicitously
in a context where the learner is conciously engaged in constructing a
\textit{public} entity ....''

\item ``... the project was not done and dropped but continued for many
weeks.  It allowed time to think, to dream, to gaze, to get a new idea
and try it and drop it or persist, time to talk, to see other people's
work and their reaction to yours---not unlike mathematics as it is for
the mathematician, but quite unlike math as it is in junior high
school.''

\end{itemize}
\end{document}

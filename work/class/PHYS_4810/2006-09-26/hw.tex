\documentclass[12pt]{article}

\usepackage{fancyhdr}
\usepackage{gensymb}
\usepackage{amsmath}

\lhead[\fancyplain]{Luke Palmer}
\chead[\fancyplain]{2006-09-18}
\rhead[\fancyplain]{PHYS 4810}
\pagestyle{fancyplain}

\begin{document}
\subsection*{diSessa, Andrea: Changing Minds: Computers, Learning, and Literacy}

Intuitive knowledge is important, productive, and (eventually)
leveragable.  The first part of this paper talks about the structure of
people's intuitions about the world, using an abstraction called
\textit{p-prims} (phenomenological primitives\footnote{p24s as they
would say in the i18n biz}).  These are visual and kinesthetic rules, or
more accurately, heuristics about how the world works.  

The second part of the paper talks about ``activities'', in particular
the fabric that seemingly irrelevant and spurious activities shares with
``genuine learning.''  This part of the paper seems to be drawing on
material covered elsewhere, and is largely anecdotal.  I think that this
part of the paper is emphasizing the importance of personalizing the
learning experience, which will forge an intuition about a foreign
concept simply by interacting with the concept.

\textit{Questions/Comments:}
\begin{itemize}
\item ``Ohm's law is easy to learn, I believe, because people see it
naturally as a simple example of Ohm's p-prim.''  Ohm's law is easy to
learn?  Wasn't Ohm's law the only physics McDermott's study tested?
\item I find that the discussion toward the end of the paper about the
dinosaur pictures is relevant to the electronics constructionist
activity discussed on Thursday.  Even though some activities may not be
directly educational, they help to connect students to the subject---and
in these cases to science in general---so that they will be motivated
and excited to learn about it on their own.
\item It is interesting how p-prims are introduced: she begins with a
few examples and discussion, so that when we see the full-length
definition, we have some context in which to put the words.  She
mentions that continuity of activity must be balanced with ``that
familiar old admonition to sequence ideas properly.''  Also, as I begin
to learn model theory, I'm starting with the compactness theorem and the
Lo\"wenheim-Skolem theorem rather than starting with the basic
precise set-theoretic definitions, so that the reasons for defining
things the way that they are will be more obvious; i.e. I am jumping
around in the book.  How important is the sequence of ideas?  More
specifically, why do we always think we must define terms before we use
them?
\end{itemize}

\subsection*{Redish: Chapter 2} 
(The rest of) this chapter was about ``bridging'', cartoons, and another
take on goals of physics education.  Bridging---taking intuitions and
building them into a correct understanding---although given little time,
was the most fascinating topic of this chapter.  That could be because
of my exposure to the tutorials which are largely conflict-model.  If a
new model is to be built, a process that takes time (as noted in the
chapter), bridging seems to be the fastest way to do that, so we can
perhaps assess how it is working in the same semester.  This chapter
also covered cartooning---taking a physical situation and ignoring (or
assuming/approximating away) the irrelevant details, and focusing on the
important ones.  This is a \textit{skill} that is seldom traditionally
taught, but usually assumed to be present.

\textit{Questions:}
\begin{itemize}
\item ``Some may think they have trouble with one or more of our
representations and actively avoid thinking about them.''  So, say there
is a concept test in class that provides information using a such a
representation; what does the student do to answer the question?
\item ``Students often see ... the drawing of a graph as the solution to
a problem ... instead of a tool to help them understand and solve a more
complex problem.''  In my proof-based courses, I am often given
statements which have tough proofs (which by themselves would evade my
solution), together with a paragraph beginning with ``\textit{Hint:}''.
The hint describes perhaps a lemma I could prove first, a function to
consider, even a representation I could draw.  Such problems usually
take me an hour to complete\footnote{Usually after having read the
problem and passively mulling about it while walking home.}, even though
the solution ends up only being a paragraph or so.  Are using questions
like this practical in an introductory course (or would they tend to
discourage more than help)?  If so, they may be a good way to teach the
importance of representations as tools.
\end{itemize}
\end{document}

\documentclass[12pt]{article}

\usepackage{fancyhdr}
\usepackage{gensymb}
\usepackage{amsmath}

\lhead[\fancyplain]{Luke Palmer}
\chead[\fancyplain]{2006-10-03}
\rhead[\fancyplain]{PHYS 4810}
\pagestyle{fancyplain}

\begin{document}
\subsection*{Elby, Andrew: What students' learning of representations
tells us about constructivism}

\textit{Questions:}
\begin{itemize}
\item I'm struggling to come up with an example where misconceptions
constructivism predicts more than fine-grained constructivism.  I.e.,
both theories say ``there are some who are consistently right, and some
who are consistently wrong'', but fine-grained constructivism says
something about people in the middle.  Is it a strictly stronger theory
than misconceptions constructivism?
\item `$x$ means $x$' means `$x$ means $x$'?  I still don't know what
$x$ means $x$ means.  I think I understand the WYSIWYG concept, but the
``boundary means boundary'' reading still evades me.
\end{itemize}
\end{document}

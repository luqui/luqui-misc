\documentclass[12pt]{article}

\usepackage{fancyhdr}
\usepackage{gensymb}
\usepackage{amsmath}
\usepackage{epsfig}

\lhead[\fancyplain]{Luke Palmer}
\chead[\fancyplain]{2006-12-14}
\rhead[\fancyplain]{PHYS 4810}
\pagestyle{fancyplain}

\begin{document}
\noindent {\large \textit{Philosophy and Approach to Teaching}}

\vspace{4ex}

I care most about whether my students are learning.   It matters a
little to me what they learn, be it a solid metaphor for voltage in
physics or a good programming practice in computer science.  But what
matters most to me is whether they are learning, thinking about what
they are learning, and continually developing and changing their theory
of the subject matter.

This philosophy realizes itself in physics as a tendency to ask
exploratory questions, mostly about abstractions and metaphors (probably
because that's what \textit{I} find interesting).  In computer science,
it realizes itself as a tendency to ask questions about efficiency: what
can you do to eliminate some repetitive code, to make something easier
to modify, etc.  I dig deeper and spend more time with a student when
the student is either already exicted about the subject (must encourage
that) or when I get the feeling that a student is trying to traverse the
path of least resistance to get a particular grade and then shutting of
his brain.
\end{document}

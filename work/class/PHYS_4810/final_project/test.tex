\documentclass[12pt]{article}

\begin{document}

Assume that all programs begin with the following:
\begin{verbatim}
    #include <iostream>
    using namespace std;
\end{verbatim}

Also, if code wouldn't compile as-is, but would if it were contained
within the \texttt{main()} function, assume that it is in the
\texttt{main()} function.

\begin{enumerate}

% This problem tests whether students make a distinction between
% characters which are letters and the newline character.
\item Consider the following code:
  \begin{verbatim}
    void out(char c) 
    {
        cout << c;
    }
    int main() 
    {
        out(?);
        out(?);
        out(?);
        return 0;
    }
  \end{verbatim}

  Is it possible to fill in the ``?''s above in order to make this
  program output the following:
  \begin{verbatim}
    a
    B
  \end{verbatim}

  If so, what would you fill in for the ``?''s in order to make it do
  that?  If not, why not?
  \newpage

% This problem tests basic understanding of function calls with return
% values.
\item Consider the following code:
  \begin{verbatim}
    double getit() {
        double r;

        do {
            cin >> r;
        } while (r < 0);
      
        return r;
    }
    int main()
    {
        double num;
        do {
            num = getit();
            cout << num << endl;
        } while (num != 0);
        return 0;
    }
  \end{verbatim}

  Describe in a sentence or two what this program does.  \bigskip
  \bigskip

  What is the output of this program when given the following as input:
  \begin{verbatim}
    3
    -1.71
    21.4
    -4
    0
  \end{verbatim}
  \newpage

% This problem tests whether students know what 'const' does.  This is
% motivated by the students frantically putting 'const's all over their
% program because the grader marks down when they don't use it.
\item The following code does not compile (the line numbers are not part
of the program):
  \begin{verbatim}
1   int main() 
2   {
3       const int multiplier = 1;
4       double input;
5       cin >> input;
6       multiplier = 2;
7       cout << multiplier * input << endl;
8       return 0;
9   }
  \end{verbatim}

  Why not?
\end{enumerate}
  
\end{document}

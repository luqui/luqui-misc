\documentclass[12pt]{article}
\usepackage{amsmath}
\usepackage{amssymb}

\newcommand{\XXX}[1]{\textbf{XXX: #1}}
\newcommand{\carddom}{\mathbb{N}_{52}}

\title{Secure Cryptographic Poker Protocol}
\author{Luke Palmer}

\begin{document}
\maketitle

There is significant suspicion, though no proof, of cheating in certain
online card rooms.  Not of individual players cheating---the sites try
to prevent that as much as possible---but of the card rooms themselves
cheating.  For instance, if a card room were to fix the hands such that
more straights and flushes are seen, then it increases the aggresive
action in the game and makes bigger pots to rake.  Not to mention, it
makes the game more exciting and attracts many players, as well as
knocking players out on ``bullshit'', emotionally encouraging them to
buy back in.

\XXX{Introduction and goals}

\section{Sketch of the Protocol}

The following pieces need to be defined before this protocol can be used
in practice.  Let $N$ be the number of players.  We use the term
\textit{publish} to mean ``send to all players''.  We use the notation
$\mathbf{x}$ to mean the vector $\langle x_i \rangle$ ranging over each
valid $i$.

\begin{itemize}
\item A cryptographic hash function to a set $\mathbb{H}$, $H:
\mathcal{P}(\carddom) \mapsto \mathbb{H}$.  It should be  ``very hard''
to determine an $x$ that will produce a given $H(x)$.  Additionally, it
should be hard to find $x$ and $y$ such that $H(x) = H(y)$.
\item A combinator function $C: \carddom^N \mapsto \carddom$ such that
if $x_i \not= y_i$ for every $i$, then $C(\mathbf{x}) \not=
C(\mathbf{y})$.  The combinator function should be a one-to-one
correspondence in each of the components of the vector; that is, if
$C(\mathbf{x}) = C(\mathbf{y})$ and there is some $j$ for which $x_j
\not= y_j$, then there must be some $k \not= j$ with $x_k \not= y_k$ as
well.
\end{itemize}

Choose one of the players to \textit{host} the hand.  It makes sense to
have the current dealer do this, but any player will do.  This player
will be responsible for generating unique random numbers in $\carddom$.
At the beginning of the hand, each player $i$ chooses a permutation
$P_i: \carddom \mapsto \carddom$ and publishes $H(P_i)$.  

To deal a community card, the host generates a unique random number $n$.
Then each player $i$ publishes his $P_i(n)$.  Each player recovers
the community card by calculating $C(\textbf{P}(n))$.

Dealing a hole card to a player $p$ proceeds in a similar way.  The host
generates a unique random number $n$.  Then each player $i$ calcuates
$P_i(n)$, and they all publish it \textit{except} for $p$.  $p$ can then
find the value of his hole card by computing $C(\textbf{P}(n))$.

The game proceeds as normal, dealing hole and community cards as they
are needed.  If there is a showdown at the end, each player $i$ who is
still in the hand publishes his $P_i$ and his hole cards.  All players
check that the published $H(P_i)$ was correct, and that his hole cards
and all published $P_i(n)$s were consistent with that permutation.  If
there is an inconsistency, then that player has cheated and appropriate
action should be taken.  If there is no showdown, then there is no
consistency checking and the hand is assumed good.

\end{document}
